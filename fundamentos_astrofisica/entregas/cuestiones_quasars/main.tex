% How to use writeLaTeX: 
%
% You edit the source code here on the left, and the preview on the
% right shows you the result within a few seconds.
%
% Bookmark this page and share the URL with your co-authors. They can
% edit at the same time!
%
% You can upload figures, bibliographies, custom classes and
% styles using the files menu.
%
% If you're new to LaTeX, the wikibook is a great place to start:
% http://en.wikibooks.org/wiki/LaTeX
%
\documentclass{tufte-handout}

%\geometry{showframe}% for debugging purposes -- displays the margins

\usepackage{amsmath}
\usepackage{siunitx}

% Set up the images/graphics package
\usepackage{graphicx}
\setkeys{Gin}{width=\linewidth,totalheight=\textheight,keepaspectratio}
\graphicspath{{graphics/}}

\title{Fundamentos de Astrofísica\\ Active Galaxies, Quasars, and Supermassive Black Holes }
\author[Alberto Garcia-Garcia]{Alberto Garcia-Garcia < agg180 [at] alu.ua.es >}
\date{\today}  % if the \date{} command is left out, the current date will be used

% The following package makes prettier tables.  We're all about the bling!
\usepackage{booktabs}

% The units package provides nice, non-stacked fractions and better spacing
% for units.
\usepackage{units}

\usepackage[americanvoltage]{circuitikz}
\usepackage{tikz}
\usetikzlibrary{arrows,shapes,positioning}
\usetikzlibrary{decorations.markings}

% The fancyvrb package lets us customize the formatting of verbatim
% environments.  We use a slightly smaller font.
\usepackage{fancyvrb}
\fvset{fontsize=\normalsize}

% Small sections of multiple columns
\usepackage{multicol}

% Provides paragraphs of dummy text
\usepackage{lipsum}

\usepackage[spanish]{babel}
\usepackage[utf8]{inputenc}

% These commands are used to pretty-print LaTeX commands
\newcommand{\doccmd}[1]{\texttt{\textbackslash#1}}% command name -- adds backslash automatically
\newcommand{\docopt}[1]{\ensuremath{\langle}\textrm{\textit{#1}}\ensuremath{\rangle}}% optional command argument
\newcommand{\docarg}[1]{\textrm{\textit{#1}}}% (required) command argument
\newenvironment{docspec}{\begin{quote}\noindent}{\end{quote}}% command specification environment
\newcommand{\docenv}[1]{\textsf{#1}}% environment name
\newcommand{\docpkg}[1]{\texttt{#1}}% package name
\newcommand{\doccls}[1]{\texttt{#1}}% document class name
\newcommand{\docclsopt}[1]{\texttt{#1}}% document class option name

\begin{document}

\maketitle% this prints the handout title, author, and date

\section{Figuring For Yourself}

\subsection{\textbf{If a quasar has a redshift of 3.3, at what fraction of the speed of light is it moving away from us?}}

En el caso relativista (el cuásar se mueve a velocidades cercanas a la velocidad de la luz), la fórmula del Doppler \emph{shift} toma la siguiente forma:

\begin{equation}
\frac{v}{c} = \frac{(z + 1)^2 - 1}{(z + 1)^2 + 1}~.
\end{equation}

El enunciado nos dice que el cuásar en cuestión tiene un \emph{redshift} de $3.3$ por lo que podemos afirmar que $z = 3.3$ y por lo tanto:

\begin{equation}
\frac{v}{c} = \frac{(3.3 + 1)^2 -1}{(3.3 + 1)^2 + 1} = \frac{17.49}{19.49} = 0.897~,
\end{equation}

por lo que el cuásar se distancia de nosotros a una velocidad del $89.7\%$ de la velocidad de la luz.

\subsection{\textbf{If a quasar is moving away from us at v/c = 0.8, what is the measured redshift?}}

De igual manera que antes:

\begin{align}
0.8 = \frac{(z + 1)^2 - 1}{(z + 1)^2 + 1}
\end{align}

\begin{align}
0.8 [(z + 1)^2 + 1] = (z + 1)^2 - 1 \rightarrow 0.8 (z + 1)**2 + 0.8 = (z + 1)**2 - 1
\end{align}

\begin{align}
0.2 (z + 1)**2 = 1.8 \rightarrow (z + 1)**2 = 9.0
\end{align}

\begin{align}
z + 1 = 3 \rightarrow z = 2
\end{align}

\bibliography{sample-handout}
\bibliographystyle{plainnat}



\end{document}