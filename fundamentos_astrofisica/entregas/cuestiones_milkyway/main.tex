% How to use writeLaTeX: 
%
% You edit the source code here on the left, and the preview on the
% right shows you the result within a few seconds.
%
% Bookmark this page and share the URL with your co-authors. They can
% edit at the same time!
%
% You can upload figures, bibliographies, custom classes and
% styles using the files menu.
%
% If you're new to LaTeX, the wikibook is a great place to start:
% http://en.wikibooks.org/wiki/LaTeX
%
\documentclass{tufte-handout}

%\geometry{showframe}% for debugging purposes -- displays the margins

\usepackage{amsmath}
\usepackage{siunitx}

% Set up the images/graphics package
\usepackage{graphicx}
\setkeys{Gin}{width=\linewidth,totalheight=\textheight,keepaspectratio}
\graphicspath{{graphics/}}

\title{Fundamentos de Astrofísica\\ The Milky Way Galaxy}
\author[Alberto Garcia-Garcia]{Alberto Garcia-Garcia < agg180 [at] alu.ua.es >}
\date{\today}  % if the \date{} command is left out, the current date will be used

% The following package makes prettier tables.  We're all about the bling!
\usepackage{booktabs}

% The units package provides nice, non-stacked fractions and better spacing
% for units.
\usepackage{units}

\usepackage[americanvoltage]{circuitikz}
\usepackage{tikz}
\usetikzlibrary{arrows,shapes,positioning}
\usetikzlibrary{decorations.markings}

% The fancyvrb package lets us customize the formatting of verbatim
% environments.  We use a slightly smaller font.
\usepackage{fancyvrb}
\fvset{fontsize=\normalsize}

% Small sections of multiple columns
\usepackage{multicol}

% Provides paragraphs of dummy text
\usepackage{lipsum}

\usepackage[spanish]{babel}
\usepackage[utf8]{inputenc}

% These commands are used to pretty-print LaTeX commands
\newcommand{\doccmd}[1]{\texttt{\textbackslash#1}}% command name -- adds backslash automatically
\newcommand{\docopt}[1]{\ensuremath{\langle}\textrm{\textit{#1}}\ensuremath{\rangle}}% optional command argument
\newcommand{\docarg}[1]{\textrm{\textit{#1}}}% (required) command argument
\newenvironment{docspec}{\begin{quote}\noindent}{\end{quote}}% command specification environment
\newcommand{\docenv}[1]{\textsf{#1}}% environment name
\newcommand{\docpkg}[1]{\texttt{#1}}% package name
\newcommand{\doccls}[1]{\texttt{#1}}% document class name
\newcommand{\docclsopt}[1]{\texttt{#1}}% document class option name

\begin{document}

\maketitle% this prints the handout title, author, and date

\section{Figuring For Yourself}

\subsection{\textbf{Assume that the Sun orbits the Center of the Galaxy at a speed of $220~[km/s]$ and a distance of $26000~[Ly]$ from the center. a) Calculate the circumference of the Sun's orbit, assuming it to be approximately circular. b) Calculate the Sun's period, the galactic year. Does it agree with the number we gave above?}}

Asumiendo que un año luz $[Ly]$ equivale a aproximadamente $9.46 \cdot 10^{12}~[km]$, el Sol se encuentra a una distancia $r = 26000 \cdot 9.46 \cdot 10^{12} = 2.46 \cdot 10^{17}~[km]$. La circunferencia de la órbita del Sol al centro de la Galaxia tiene una longitud de $l = 2 \pi r = 2 \pi \cdot 2.46 \cdot 10^{17} = 1.54 \cdot 10^{18} ~ [km]$.

El período (el año galáctico) se define como el tiempo necesario para completar una órbita $ t = l / v$. Siendo la velocidad orbital del Sol $220~[km/s]$ y la longitud de su órbita $l = 1.54 \cdot 10^{18}~[km]$, el tiempo que tarda el Sol en realizar una revolución alrededor del centro galáctico es de $t = 1.54 \cdot 10^{18} / 220 = 7.02 \cdot 10^{15}~[s]$.

En el libro se afirma lo siguiente:

\begin{quotation}
  The Sun, like all the other stars in the Galaxy, orbits the center of the Milky Way. Our star’s orbit is nearly circular
and lies in the Galaxy’s disk. The speed of the Sun in its orbit is about 200 kilometers per second, which means
it takes us approximately 225 million years to go once around the center of the Galaxy.
\end{quotation}

Según nuestros cálculos, el Sol tarda $7.02 \cdot 10^{15}~[s]$, o lo que es lo mismo $224.75~[Myr]$ (número que casa con el proporcionado en el libro).

\bibliography{sample-handout}
\bibliographystyle{plainnat}



\end{document}