% How to use writeLaTeX: 
%
% You edit the source code here on the left, and the preview on the
% right shows you the result within a few seconds.
%
% Bookmark this page and share the URL with your co-authors. They can
% edit at the same time!
%
% You can upload figures, bibliographies, custom classes and
% styles using the files menu.
%
% If you're new to LaTeX, the wikibook is a great place to start:
% http://en.wikibooks.org/wiki/LaTeX
%
\documentclass{tufte-handout}

%\geometry{showframe}% for debugging purposes -- displays the margins

\usepackage{amsmath}
\usepackage{siunitx}

% Set up the images/graphics package
\usepackage{graphicx}
\setkeys{Gin}{width=\linewidth,totalheight=\textheight,keepaspectratio}
\graphicspath{{graphics/}}

\title{Fundamentos de Astrofísica\thanks{Inspired by Edward~R. Tufte!}}
\author[Alberto Garcia-Garcia]{Alberto Garcia-Garcia}
\date{\today}  % if the \date{} command is left out, the current date will be used

% The following package makes prettier tables.  We're all about the bling!
\usepackage{booktabs}

% The units package provides nice, non-stacked fractions and better spacing
% for units.
\usepackage{units}

\usepackage[americanvoltage]{circuitikz}
\usepackage{tikz}
\usetikzlibrary{arrows,shapes,positioning}
\usetikzlibrary{decorations.markings}

% The fancyvrb package lets us customize the formatting of verbatim
% environments.  We use a slightly smaller font.
\usepackage{fancyvrb}
\fvset{fontsize=\normalsize}

% Small sections of multiple columns
\usepackage{multicol}

% Provides paragraphs of dummy text
\usepackage{lipsum}

\usepackage[spanish]{babel}
\usepackage[utf8]{inputenc}

% These commands are used to pretty-print LaTeX commands
\newcommand{\doccmd}[1]{\texttt{\textbackslash#1}}% command name -- adds backslash automatically
\newcommand{\docopt}[1]{\ensuremath{\langle}\textrm{\textit{#1}}\ensuremath{\rangle}}% optional command argument
\newcommand{\docarg}[1]{\textrm{\textit{#1}}}% (required) command argument
\newenvironment{docspec}{\begin{quote}\noindent}{\end{quote}}% command specification environment
\newcommand{\docenv}[1]{\textsf{#1}}% environment name
\newcommand{\docpkg}[1]{\texttt{#1}}% package name
\newcommand{\doccls}[1]{\texttt{#1}}% document class name
\newcommand{\docclsopt}[1]{\texttt{#1}}% document class option name

\begin{document}

\maketitle% this prints the handout title, author, and date

\begin{abstract}
%
\end{abstract}

\tableofcontents

\clearpage

\section{Constantes}

$1 [\si{\angstrom}] = 0.1 ~[nm]$\\
$\sigma = 1.380 \cdot 10^{-23} [J \cdot K^{-1}]$\\
$\sigma_b = 8.617 \cdot 10^{-5} [eV \cdot K^{-1}]$\\
$\sigma = 1.380 \cdot 10^{-16} [erg \cdot K^{-1}]$\\ 

\clearpage

\section{Astronomía de Posición e Instrumentación Astronómica}

\subsection{Astronomía de Posición}

\subsection{Telescopios}

Existen tres components básicos en cualquier sistema moderno para medir la radiación proveniente de fuentes astronómicas: telescopios, dispositivos acoplados al telescopio que ordenan la radiación capturada según su longitud de onda y por último algún tipo de detector sensible a la radiación en las longitudes de onda determinadas.

Las funciones más importantes de un telescopio son: (1) recoger la luz "débil" de una fuente astronómica y (2) concentrar toda ella en un único punto o imagen. Dado que la mayoría de los objetos astronómicos estudiados son muy débiles, cuanta más luz se pueda recoger, mejor podemos estudiarlos (de ahí que los telescopios hayan crecido en tamaño continuamente).

Los telescopios utilizan una lente o un espejo para recoger la luz y en todos ellos la capacidad de absorción de luz está determinada por el área del dispositivo que la captuar. Dado que la mayoría utilizan los mismos dispositivos, su capacidad de captura de luz la podemos comparar mediante su apertura o diámetro.

\subsection{Instrumentación}

\clearpage

\section{Astrofisica Estelar}

\subsection{Luminosidad y Brillo}

La luminosidad $L~[erg \cdot s^{-1}]$ es la energía total emitida por unidad de tiempo mientras que el brillo $l~[erg \cdot s^{-1} \cdot m^{-2}]$ es la energía recibida por área y unidad de tiempo (es decir, el flujo de energía).

\marginnote{Problema 1. La constante solar es la cantidad de energía recibida en la parte externa de la atmósfera terrestre en forma de radiación solar, por unidad de tiempo y de superficie, y medida en un plano perpendicular a los rayos del Sol. El valor es de $1365~[W/m^2]$. Estima a qué distancia debemos colocar una bombilla de $100~[W]$ para que su flujo sea igual a la constante solar.\\

\begin{align}
  r = \sqrt{\frac{L}{4\pi l}} = \sqrt{\frac{100}{4\pi 1365}} = 0.076 [m]
\end{align}
}

Ambas magnitudes se relacionan de la siguiente forma: imaginemos una estrella de luminosidad $L$ rodeada por una esfera de radio $r$. Asumiendo que la luz no es absorbida durante su viaje hacia la esfera, el flujo $l$ medido a una distancia $r$ es inversamente proporcional al cuadrado de la distancia (\emph{inverse square law of light}):

\begin{align}
  l = \frac{L}{4\pi r^2}
\end{align}

\subsection{Escala de Magnitudes Estelares}

Utilizando la \emph{inverse square law}, podemos asignar una magnitud absoluta $M$ a cada estrella, la cual se define como la magnitud aparente $m$ de una estrella si se encontrara a $10 [pc]$ de distancia.

\marginnote{
Una diferencia de 5 magnitudes aparentes supone que la estrella de menor magnitud es 100 veces más brillante que la estrella de mayor magnitud, por lo que podemos expresar el ratio de sus flujos como \\\ \\

$\frac{l_0}{l} = 100^{(m_1 - m_2)/5}$\\\ \\

por lo que tomando logaritmos\\\ \\

$m_1 - m_2 = -2.5\log_{10}\frac{l}{l_0}$
}

\begin{align}
  m = -2.5 \log{\frac{l}{l_0}}
\end{align}

\begin{align}
  M = -2.5 \log{\frac{L}{L_0}}
\end{align}

Podemos de esta forma establecer una conexión entre la magnitud aparente, la magnitud absoluta y la distancia de una estrella:

\begin{align}
  100^{(m - M)/5} = \frac{l_{10}}{l} = (\frac{d}{10 [pc]})^2
\end{align}

Podemos por lo tanto determinar la distancia como

\begin{align}
  d = 10^{(m - M + 5)/5} [pc]
\end{align}

\marginnote{Tomando logaritmos podemos establecer el \emph{distance modulus}, es decir, la cantidad $m - M$:\\\ \\

$m - M = 5 \log_{10}(d) - 5 = 5 \log_{10}\frac{d}{10}$
}

De estas ecuaciones se desprende que para dos estrellas a la misma distancia, el ratio de su brillo es igual al ratio de sus luminosidades. Por lo tanto, tomando el Sol como una de esas estellas podemos obtener la relación entre la magnitud absoluta de una estrella y su luminosidad, así como entre su magnitud relativa y su brillo:

\begin{align}
  M = M_{\odot} - 2.5 \log_{10}\frac{L}{L_{\odot}}
\end{align}

\begin{align}
  m = M_{\odot} - 2.5 \log_{10}\frac{l}{l_{10, \odot}}
\end{align}

\subsection{Temperaturas Estelares}

Todos los objetos con temperatura por encima del cero absoluto emiten luz de todas las longitudes de onda con diferentes niveles de eficiencia. Un emisor ideal es aquél que absorbe toda la luz incidente y radia toda esta energía con un espectro denominado espectro de cuerpo negro. Dado que un emisor ideal no refleja luz se le conoce como cuerpo negro (\emph{blackbody}) y la radiación que emite se conoce como radiación de cuerpo negro. Las estrellas son cuerpos negros (por lo menos de forma aproximada).

\paragraph{Ley de Wien}

Un cuerpo negro de temperatura $T$ emite un espectro continua con cierta energía en cada longitud de onda de manera que existe un pico máximo de energía en la longitud de onda $\lambda_{max}$ . La relación entre esta longitud de onda máxima y la temperatura fue descubierta por Wien:

\begin{align}
  \lambda_{max} T = 0.290 [cm \cdot K]
\end{align}

\begin{align}
  \lambda_{max} T = (5000~[\si{\angstrom}])(5800~[K])
\end{align}

\marginnote{
  
\emph{Problema 2.} Expresa la temperatura ambiente (300 $K$) en $[eV]$ y calcula a qué temperatura corresponde una energía de 13.6 $[ev]$.\\\ \\

La temperatura y la energía se relacionan mediante la constante de Boltzmann $E = \sigma T$.\\\ \\

Una temperatura de $300 [K]$ corresponde a una energía $ E = \sigma T = 8.617 \cdot 10^{-5} \cdot 300 = 0.0259 [eV]$.\\\ \\

Por otro lado, una energía de $13.6 [eV]$ corresponde a una temperatura $T = E / \sigma = 13.6 / 8.617 \cdot 10^{-5} = 1.58 \cdot 10^{15} [K]$.\\

\rule{\linewidth}{0.4pt}
}

\paragraph{Ley de Boltzmann}

A medida que la temperatura de un cuerpo negro aumenta, también lo hace su energía emitida por segundo en todas las longitudes de onda. Este fenómeno, descubierto por el físico austríaco Josef Stefan Boltzmann se expresa en función del área de dicho cuerpo negro $A$ como:

\begin{align}
  L = A \sigma T^4 ~ [erg \cdot s^{-1}]
\end{align}

Combinando esta ecuación con el brillo, podemos determinar la temperatura efectiva $T_e$ de la superficie de una estrella:

\begin{align}
  l = \sigma T_e^4 [erg \cdot s^{-1} \cdot m^{-2}]
\end{align}

\marginnote{
La constante de Boltzmann relaciona temperatura absoluta y energía. Toma los siguientes valores:\\

$\sigma = 1.380 \cdot 10^{-23} [J \cdot K^{-1}]$\\
$\sigma = 8.617 \cdot 10^{-5} [eV \cdot K^{-1}]$\\
$\sigma = 1.380 \cdot 10^{-16} [erg \cdot K^{-1}]$\\ 
}

\subsection{Color y Temperatura}

\marginnote{
\emph{Problema 3.} Sirio es la estrella más brillante del cielo nocturno. Su paralaje es $p = 0.377 ["]$ y sus magnitudes aparentes son $U = -1.50$, $B = -1.46$ y $V = -1.46$.\\\ \\

La distancia a Sirio es por lo tanto $d = 1 / p = 1 / 0.377 = 2.65 [pc]$. Dicha cantidad equivale a $d = 2.65 \cdot 3.26 = 8.65 [ly]$ y $d = 2.65 \cdot 206265 = 5.44 \cdot 10^{15} [AU]$ o $d = 8.14 \cdot 10^{16} [cm]$.\\\ \\

Los índices de color de Sirio son $U-B = -0.04$ y $B-V = 0$. A partir de ellos podemos estimar una temperatura de $9900~[K]$ (tomando cualquier diagrama que relacione estos índices de color con la temperatura) o empleando la fórmula de Ballesteros:

$T = 4600 \left(\frac{1}{0.92(B-V) + 1.7} + \frac{1}{0.92(B-V) + 0.62}\right) \simeq 10000~[K] = 10000 \cdot \sigma_b = 0.86~[eV]$\\\ \\

La relación módulo-distancia $m - M = 5\log_{10}(d) - 5$ nos permite obtener las magnitudes absolutas cada una de las bandas en función de sus magnitudes relativas $U$, $B$, $V$ ya que $M = m + 2.88$ siendo $d = 2.65~[pc]$:\\

$M_U = -1.50 + 2.88 = 1.38$\\
$M_B = -1.46 + 2.88 = 1.42$\\
$M_V = -1.46 + 2.88 = 1.42$\\

}

Las magnitudes absolutas y aparentes medidas sobre todas las longitudes de onda de la luz emitida por una estrella se conocen como \emph{magnitudes bolométricas} $m_{bol}$ y $M_{bol}$. En la práctica, los detectores son sensibles a una cierta región de longitudes de onda. Así pues, el color de una estrella puede ser determinado con precisión empleando filtros que transmitan la luz de la estrella únicamente en ciertos rangos de longitudes de onda relativamente estrechos. Es el caso del sistema $UBV$ segundoún el cual la magnitud aparente de una estrella es determinada a partir de tres filtros:

\begin{itemize}
  \item U, la magnitud ultravioleta de la estrella (filtro centrado en $3650 [\si{\angstrom}]$ y con un ancho de banda de $680 [\si{\angstrom}]$).
  \item B, la magnitud azul de la estrella (filtro centrado en $4400 [\si{\angstrom}]$ y con un ancho de banda de $980 [\si{\angstrom}]$).
  \item V, la magnitud visual de la estrella (filtro centrado en $5500 [\si{\angstrom}]$ y con un ancho de banda de $890 [\si{\angstrom}]$).
\end{itemize}

Las magnitudes absolutas de color de la estrella $M_U$, $M_B$ y $M_V$ pueden determinarse si la distancia $d$ es conocida. Para ello, recurrimos a los índices de color $(U-B)$ y $(B-V)$:

\begin{align}
  U - B = M_U - M_B
\end{align}
\begin{align}
  B - V = M_B = M_V
\end{align}

\marginnote{
\emph{Problema 4.} Usando una temperatura efectiva para la estrella Vega de $9500~[K]$ y suponiendo que tenemos tres filtros de badna U,B,V con los siguientes valores de longitud de onda central y ancho de banda, determina en qué filtro se verá más brillante. $\lambda_U = 365~[nm]$, $\Delta\lambda_U = 68~[nm]$, $\lambda_B = 440~[nm]$, $\Delta\lambda_B = 98~[nm]$, $\lambda_V = 550~[nm]$, $\Delta\lambda_V = 89~[nm]$.\\\ \\

De acuerdo a la Ley de Wien, el espectro de Vega tiene un pico en la longitud de onda $\lambda_{max} = \frac{5000~[\si{\angstrom}] 5800~[K]}{9500~[K]} = 3052.63~[\si{\angstrom}] \approx 305~[nm]$, lo cual corresponde al espectro ultravioleta.
}

\subsection{Distancia a las Estrellas}

La distancia en pársecs a una estrella se puede determinar por su paralaje en segundos:

\begin{align}
  d = 1 / p [pc]
\end{align}

\clearpage

\section{Estructura y Evolución Estelar}

\subsection{De Secuencia Principal a Gigantes Rojas}

\begin{quotation}
  Chapter 22. Stars from Adolescen to Old Age. Section 22.1 Evolution from the Main Sequence to Read Giants
\end{quotation}

\marginnote{Dado que el hidrógeno es el componente más abundante, la secuencia principal es el estado en el que pasan la mayor parte de su vida.}

Una vez la estrella ha alcanzado la secuencia principal, deriva mayormente su energía de la conversión de hidrógeno a helio mediante fusión nuclear en el núcleo de la estrella.

El momento en el que la estrella deja de contraerse y comienza a fusionar hidrógeno se denomina \emph{zero-age main sequence}.

\marginnote{La temperatura y la densidad en el interior de la estrella se incrementan paulatinamente a medida que el helio se acumula; con el incremento de temperatura, los protones adquieren más energía de movimiento y por lo tanto el ratio de fusión aumenta por lo que también lo hace la energía generada y por lo tanto la luminosidad de la estrella.}

Dado que únicamente el $\approx 1\%$ del hidrógeno de la reacción de fusión se convierte en energía, la masa total de la estrella no cambia significativamente pero sí lo hace su composición química lo cual cambia su luminosidad, temperatura, tamaño y también la estructura interior de la misma.

\marginnote{Las estrellas masivas son las más luminosas pero también tienen un ciclo de vida mucho más corto que las menos masivas.}

\paragraph{Tiempo de Vida en Secuencia Principal} la cantidad de años que una estrella permanece en secuencia principal depende de su masa (cuánta cantidad de combustible nuclear tiene) y cómo de rápido lo consume. Las estrellas más masivas consumen combustible mucho más rápido que las estrellas poco masivas.

\marginnote{Las estrellas dedican aproximadamente un $90\%$ de su vida en secuencia principal y dependiendo de su masa esta puede durar desde $1$ millón hasta $200$ billones de años.}

Aunque parezca contraintuitivo, la cantidad de combustible consumida depende de la temperatura en el núcleo de la estrella lo cual depende de su masa. El peso de las capas superiores de la estrella determina la presión en el núcleo. Estrellas masivas necesitan una presión elevada en el núcleo para balancearlo. Esa alta presión se produce por una elevada temperatura y a mayor temperatura mayor consumo de combustible.

\paragraph{De Secuencia Principal a Roja Gigante} eventualmente, el hidrógeno se consume y el núcleo únicamente contiene helio. La temperatura del núcleo sin embargo no es lo suficientemente elevada para fusionar helio y por lo tanto no hay reacción nuclear. En este momento, la gravedad gana la partida de nuevo y el núcleo se contrae y la energía del material cayendo hacia adentro se convierte en calor, la gravedad pasa a ser la fuente principal de energía de la estrella.

El calor generado fluye hacia el exterior y calienta las capas de hidrógeno fuera del núcleo y que no fue capaz de alcanzar la temperatura necesaria para la fusión anteriormente ahora sí activada.

\marginnote{A medida que la estrella se expande se convierte en más luminosa gracias a la fusión del hidrógeno en las capas exteriores pero al la vez al expandirse se enfría en la superficie y su radiación se enrojece.}

A partir de este punto se produce el efecto combinado de calor a causa de la contracción gravitatoria del núcleo y de la fusión de hidrógeno en las capas exteriores el cual expande las capas exteriores de la estrella.


\clearpage

\section{El Sol}

\subsection{Estructura y Composición}

El Sol es una esfera de gases extremadamente calientes e ionizados que brilla por su propia energía. No posee una superficie sólida como la Tierra ni tampoco un núcleo sólido. Sin embargo posee una estructura en capas.

\marginnote{La mayoría de los elementos en el sol se encuentran en forma de átomos con un pequeño número de moléculas en forma de gases. Las elevadas temperaturas impiden la formación de sólidos o líquidos y además la mayoría de los átomos están ionizados. Existe una gran cantidad de electrones libres e iones positivamente cargados en el Sol por lo que genera un entorno cargado eléctricamente. Este gas ionizado caliente suele recibir el nombre de \emph{plasma}.}

Es una estrella de clase G2V con una temperatura efectiva de $5800~[K]$ y una luminosidad de $3.8\cdot 10^{26}~[W]$. Su masa es de $1.99 \cdot 10^{30}~[kg]$ y posee un período de rotación en el ecuador de $24$ días y $16$ horas. Se encuentra alejado de nosotros a una distancia media de $1~[AU] = 149,597,892~[km]$.

\paragraph{Composición de la Atmósfera}

Empleando el espectro de absorción podemos concluir que la atmósfera solar contiene los mismos elementos que la terrestre pero en distinta proporción: el $73\%$ de su masa es hidrógeno, $25\%$ helio, el $0.20\%$ carbono y el resto diversos elementos como nitrógeno, oxígeno, neón, magnesio, silicio, sulfuro y hierro.

\paragraph{Capas no Visibles} la capa más interna del Sol es el núcleo, extremadamente denso y fuente de toda su energía. Aproximadamente el $20\%$ del tamaño total del interior del Sol, posee una temperatura de $15\cdot 10^6~[K]$.

\begin{marginfigure}
  \includegraphics[width=\linewidth]{img/sun}
  \caption{Estructura en capas del Sol.}
\end{marginfigure}

Por encima del núcleo se encuentra la zona radiativa. Aproximadamente comienza al $25\%$ de distancia desde la superficie y se extiende hasta el $70\%$. La luz generada en el núcleo se transporta muy lentamente ya que la alta densidad de esta zona impide que los fotones se muevan sin colisionar.

Le sigue la zona convectiva, aproximadamente de $200,000~[km]$ de profundidad en la que la energía de la zona radiativa se transporta a la superficie mediante células convectivas gigantes. El plasma en el límite de esta zona está muy caliente y genera burbujas en la superficie donde disipa su calor al espacio.

\paragraph{La Fotosfera} la atmósfera exterior del sol es transparente, lo cual permite el paso de la luz hasta cierto punto. La atmósfera del sol cambia de completamente transparente a prácticamente opaca en apenas $400~[km]$, esta zona se denomina fotosfera.

La fotosfera tiene una presión de aproximadamente el $10\%$ de la presión terrestre al nivel del mar y está granulada. Estos granos típicamente de $700$ a $1000~[km]$ de diámetro poseen una esperanza de vida corta ($5-10~[min]$) y están rodeados por una región oscura.

Se oscurece conforme nos acercamos al borde del círculo proyectado en un fenómeno conocido como \emph{limb darkening} debido a la profundidad óptica.

\paragraph{La Cromosfera} la región de la atmósfera que se encuentra justo por encima de la fotosfera es la cromosfera. Aproximadamente de $2000$ a $3000~[km]$ de grosor, se compone de gases calientes emitiendo luz en longitudes de onda discretas (su espectro consiste en líneas de emisión muy brillates). Su color rojizo se debe a que su emisión más fuerte es la causada por el hidrógeno. Se ecuentra alrededor de $10000~[K]$, más caliente que la fotosfera.

\begin{marginfigure}
  \includegraphics[width=\linewidth]{img/sun_t}
  \caption{Incremento de la temperatura del SOl hacia el exterior.}
\end{marginfigure}

\paragraph{La Zona de Transición}

Después de la cromosfera, la temperatura cambia hasta aumentar al millón de grados o más en una región llamada corona. La región en la que ocurre este incremento dramático, de apenas decenas de kilómetros de espesor, se denomina zona de transición.

\paragraph{La Corona}

la parte más externa de la atmósfera solar se denomina corona. Se extiende por millones de kilómetros sobre la fotosfera y emite una gran cantidad de luz aunque esta se ve eclipsada por la luz emitida por la fotosfera. Los estudios espectrales sugieren que la corona es muy poco densa y se extiende tanto en el espacio que podríamos considerar que estamos viviendo en el interior de la atmósfera del Sol.

También es la parte más caliente, llegando a alcanzar un millón de $[K]$ y por lo que los átomos están altamente ionizados y se pueden observar diversos fenómenos magnéticos.

\paragraph{El Viento Solar}

uno de los descubrimientos más importantes de la atmósfera solar es la producción de chorros o corrientes de partículas cargadas referidos como viento solar. Estas partículas fluyen del interior del sol al sistema solar aproximadamente a $400~[km/s]$ a causa de que los gases en la corona están tan calientes y se mueven tan rápido que no pueden ser contenidos por la propia gravedad del Sol.

Aunque la corona parezca completamente uniforme, imágenes de rayos X y ultravioleta muestran que la corona posee bucles y llamaradas así como regiones brillantes y oscuras. En dichas regiones, las líneas de campo magnético se extienden lejos del sol en lugar de volver a la superficie del sol. Sin la atadura del campo magnético, el viento solar escapa, eyectando material solar al espacio.

Las líneas de campo se extienden hasta nosotros aunque llegan a la Tierra en los polos norte y sur magnéticos. Allí, las partículas cargadas siguien el campo y caen a nuestra atmósfera y al chocar con las partículas de aire producen las auroras boreales.

\subsection{El Ciclo Solar}

El Sol se encuentra en perpetuo cambio. La primera evidencia de dichos cambios sobrevino por el estudio de las manchas solares (que parecen oscuras en contraste con la brillante y más caliente fotosfera).

Las manchas solares vienen y van en pocas horas o meses. Si sobreviven y se desarrollan cosisten en dos partes: umbra y penumbra. Además, no están fijos sino que se mueven lentamente a medida que el Sol rota (más rápido en el ecuador).

Atendiendo al movimiento aparente de estas manchas solares a medida que el Sol gira y se desplazan por su disco, se pudo establecer que el Sol posee un período de rotación sobre su eje de aproximadamente un mes (27 días). Sin embargo, el Sol no rota como un sólido como la Tierra, sino que sus gases se mueven a distinta velocidad dependiendo de su latitud, este comportamiento se denomina \emph{differential rotation}.

\paragraph{El Ciclo de las Manchas Solares}

El número de manchas solares cambia de manera sistemática en ciclos de aproximadamente una década de duración.

Aunque las manchas individuales viven por poco tiempo, el número total de manchas visibles es mucho mayor en ciertos momentos que en otros. Aproximadamente, cada 11 años se produce un máximo de manchas solares.

\paragraph{Magnetismo y el Ciclo Solar}

La actividad solar y su ciclo están regidos por el campo magnético cambiante del propio Sol.

\marginnote{El campo magnético del Sol se mide mediante el efecto Zeeman. El presencia de un campo magnético fuerte, los niveles de energía del átomo se separan en varios niveles cercanos. La separación de dichos niveles es proporcional a la intensidad del campo. Esto provoca que las líneas espectrales no sean líneas solas sino una serie de líneas separadas ligeramente.}

Las mediciones por efecto Zeeman del espectro de luz emitido por las manchas solares muestran que poseen campos magnéticos muy intensos. Además, las manchas solares suelen aparecer en pares norte/sur magnéticos o incluso pueden llegar a agruparse teniendo una polaridad en un hemisferio y otra en el opuesto.

Ocurre también que en el siguiente ciclo solar, la polaridad de las manchas solares predominantes se invierte en cada hemisferio. Los magnetogramas del Sol nos permiten visualizar estas situaciones.

El campo magnético del Sol como vemos, es francamente complicado, principalmente debido a su propia \emph{dynamo}. La fuente de energía cinética del sol es toda la maquinaria de gases ionizados calientes que generan corrientes eléctricas que a la vez generan campos magnéticos. El consenso general es que la fuente de esta dinamo se encuentra en la zona de convección o en la interfaz entre esta zona y la radiativa. Y, aunque tenemos un entendimiento general del comportamiento magnético del Sol, no se ha desarrollado un modelo físico que describa en detalle todos los procesos que ocurren.

\marginnote{El período durante el siglo XVIII en el que el número de manchas solares en un ciclo solar fue inusualmente bajo se conoce como el mínimo de Maunder.}

Casi el único hecho que comprendemos es que la rotación diferencial junto con la convección bajo la superficie deforman y distorsionan los campos magnéticos provocando su crecimiento y decaimiento, regenerándose con polaridad opuesta cada ciclo. Cuando los campos crecen en intensidad y se forma un máximo, fluyen hacia el exterior en forma de bucles, creando así regiones de manchas solares.

\subsection{Actividad Solar en la Fotosfera}

\begin{itemize}
  \item Manchas solares: son la forma más conocida de actividad solar. Entre $10,000$ y $100,000~[km]$, suelen aparecer por pares entre las células convectivas. La polaridad de la mancha que lidera la rotación es la misma que la polaridad de la región polar del hemisferio. Primero son brillantes (fáculas) y luego se vuelven oscuras y crecen rápidamente. A menudo entre $1500$ y $2500~[K]$ más frías que el resto de la superficie por lo que provocan una disminución en la luminosidad. Se deben a tubos de campo magnético que inhiben la convección.

  \item Protuberancias o Prominencias: formaciones gaseosas que se extienden desde la base de la cromosfera hasta la corona. Se trata de plasma con temperaturas similares a la de la cromosfera. Sigue las líneas de campo formando bucles. Se forman por recombinación de las líneas de campo.

  \item Fulguraciones o Erupciones: violentos estallidos que se producen en la cromosfera. Duran desde minutos a horas. Conducen a la expulsión de partículas cargadas y suelen ocurrir cuando campos magnéticos que apuntan en direcciones opuestas interaccionan y se destruyen el uno al otro.
  
  \item Eyecciones de Masa Coronal: suelen suceder con las fulguraciones y están asociadas con las protuberancias. La materia eyectada es plasma formado mayoritariamente por protones y electrones. Pueden tener serias consecuencias sobre el campo magnético terrestre.
\end{itemize}

\subsection{Clima Espacial}

\subsection{Energía Térmica y Gravitacional}

\subsection{Masa, Energía y la Teoría de la Relatividad}

\subsection{El Interior del Sol}

\clearpage

\section{El Sistema Solar}

Nuestro Sistema Solar contiene una estrella, ocho planetas con 210 satélites conocidos, 5 planetas enanos (aunque previsiblemente muchos más), casi un millón de cuerpos menores y asteroides, el cinturón de asteroides, el cinturón de Kuiper, la nube de Oort, más de 4000 cometas y el medio interplanetario. Los límites del Sistema Solar están marcados por la esférica nube de Oort.

Los planetas del Sistema Solar se clasifican en inferiores/superiores (dependiendo de si su órbita es mayor o menor que la de la Tierra) y en interiores/exteriores (interiores telúricois y exteriores gaseosos). Plutón se considera un planeta enano.

Los planetas interiores suelen ser pequeños, densos, sólidos y de rotación lenta mientras que los exteriores son grandes, menos densos, gasesosos y de rotación rápida.

\subsection{Mercurio}

Es el planeta más cercano al Sol ($\approx 4900~[km]$ de diámetro, $0.055~M_{\bigoplus}$); no tiene atmósfera como tal y es de una densidad elevada. Su aspecto es parecido a la Luna pero su composición es distinta. No tiene proceso regenerador de su superficie. Su año y día están en resonancia.

Posee una temperatura de hasta $700~[K]$ en el lado iluminado y $100~[K]$ en el nocturno. La sonda Messenger identificó agua helada en el fondo de los cráteres polares y una levísima atmósfera por captura del viento solar.

Puede ocasionalmente verse transitar por la superficie del sol 13 ó 14 eces cada siglo. Su máxima elongación es de 28 grados.

\subsection{Venus}

Segundo planeta ($\sim 12000~[km]$ de diámetro, $0.815~M_{\bigoplus}$), possee una atmósfera con gran efecto invernadero debido a la gran cantidad de dióxido de carbono. Está cubierto por una densa capa de ácido sulfúrico. Su superficie no está saturada de cráteres pero se desconoce su mecanismo de regeneración. Posee una rotación retrógrada y lenta.

Posee una temperatura de hasta $740~[K]$ con una gran presión, lo cual dificulta su exploración. Además, su orografía está dominada por vulcanismo.

Sus vientos giran mucho más rápido que la superficie del planeta y posee una máxima elongación de 48 grados realizando tránsitos sobre el Sol.

\subsection{Tierra}

El planeta más grande de los interiores ($\sim 12000~[km]$ de diámetro). Posee una atmósfera mayoritariamente de oxígeno ($21\%$) con agua líquida en la superficie. Un campo magnético importante y geológicamente activa.

\paragraph{La Luna}

Satélite de la Tierra (aproximadamente $3500~[km]$ de diámetro y $0.012 M_{\bigoplus}$). Sin ningún tipo de atmósfera y una superficie saturada de cráteres sin regeneración.

Como apunte, su período de rotación es igual al período orbital así que siempre muestra la misma cara a la Tierra con zoans oscuras y claras.

\subsection{Marte}

El último planeta interior (aproximadamente $6800~[km]$ de diámetro y $0.107~M_{\bigoplus}$). Atmósfera poco densa (principalmente de dióxido de carbono) y sin saturación de cráteres lo cual sugiere algún mecanismo de regeneración.

Hay evidencias de escorrentías (agua líquida en el pasado o quizás en el subsuelo), vulcanismo (¿extinto?) y tectónica de placas (¿también extinta?). Posee dos pequeños satélites capturados (Fobos y Deimos).

La temperatura media en la superficie es de $210~[K]$ con muy baja presión atmosférica, grandes tormentas de polvo y hielos de dióxido de carbono.

\subsection{El Cinturón de Asteroides}

Generalmente referido como el \emph{Main Belt} para diferenciarlo del \emph{Kuiper Belt}, se localiza entre las órbitas de Marte y Júpiter y cubre una distancia de alrededor de 2-3 $[UA]$ aunque en su mayoría consiste en espacio vacío, contiene trillones de rocas espaciales. La dinámica del mismo está dominada por las resonancias orbitales de Júpiter.

Cuatro objetos en él contienen más de la mitad de la masa total del cinturón (la cual se sitúa alrededor de un $4~\%$ de la masa de la Luna): el planeta enano Ceres, Vesta, Pallas e Hygiea.

La mayoría de los objetos del cinturón son cuerpos rocososo compuestos de minerales y son ricos en hierro y níquel por lo que la teoría de mayor calado es que sean restos de la formación del Sistema Solar: una nube enorme de gas y partículas que poco a poco fueron acretando.

La sonda \emph{Dawn} fue enviada para investigarlo, la cual orbitó Vesta ($525~[km]$) en 2011 descubriendo que se trata más bien de un protoplaneta más que de un trozo de roca. De la misma forma, se descubrió que Ceres ($939~[km]$) tiene una capa de hielo-agua en 2015. Los dos cuerpos de mayor tamaño siguientes son Pallas ($512~[km]$) e Hygiea ($434~[km]$).

\subsection{Planetas Gigantes}

Más allá de Marte y del cinturón de asteroides se sitúan los dos planetas más grandes del Sistema Solar: los planetas gaseosos Júpiter ($140000~[km]$) y Saturno ($120000~[km]$), principalmente compuestos por helio e hidrógeno (con mucha más cantidad del segundo que del primero). La principal teoría de su formación sostiene que se formaron al comienzo de la historia del Sistema Solar, lo cual les permitió acumular grandes cantidades de los dos gases más ligeros.

Ambos planetas rotan muy rápido y muestran rotación diferencial y achatamiento. Los dos poseen un sistema de anillos de polvo aunque el de Júpiter es muy tenue. El de Saturno está formado por millones de fragmentos de hielo y roca orbitando el plano ecuatorial con un grosor máximo de dos kilómetros con tres regiones principales (A, B, C más interna) seapradas por zonas menos densas (Cassini es la más grande entre A y B).

Júpiter está compuesto aproximadamente de $90~\%$ de hidrógeno y $10~\%$ de helio mientras que Saturno contiene un $96~\%$ de hidrógeno por únicamente un $3~\%$ de helio.

Mientras que Júpiter es tremendamente masivo ($300 M_{\bigoplus}$), Saturno ($95 M_{\bigoplus}$) tiene una densidad del $12\%$ respecto a la de la Tierra.

Júpiter posee el campo magnético más fuerte de todos los planetas del Sistema Solar y su radiación emitida es capaz de destruir sondas cercanas. La radiación emitida por Saturno es relativamente débil en comparación.

Aunque parezca contraintuitivo, ambos planetas poseen una superficie pese a ser gaseosos (localizada en el punto en el que la presión atmosférica es igual a la presión atmosférica en la superficie de la Tierra).

La sonda \emph{Galileo} en el año 2003 se estrelló contra la superficie de Júpiter registrando vientos de 700 $[km/h]$ y una temperatura de 151 grados centígrados justo antes de perder el contacto.

Creemos que ambos planetas poseen núcleos sólidos o al menos semi-sólidos. En cualquier caso, ambos núcleos internos están rodeados por una mezcla extraña de hidrógeno metálico líquido. Dadas las elevadas temperaturas y presiones cerca de ambos núcleos, se piensa que las moléculas de hidrógeno pueden soltar sus electrones convirtiéndose en metal líquido (causa del elevado campo magnético).

Júpiter posee cuatro lunas (aunque se conocen más de 75 satélites): Ganímedes, Callisto, Io y Europa. Ganímedes es la mayor luna del sistema solar, más grande que Mercurio. Por su parte, Callisto es el cuerpo geológicamente más activo del Sistema Solar.

La luna más importante de Saturno es Titán: la segunda mayor luna del Sistema Solar y la única con atmósfera densa. La sonda Hyugens encontró grandes lagos de metano líquido y evidencia de lluvia en una atmósfera compuesta principalmente de nitrógeno y una temperatura del orden de $95~[K]$.

\subsection{Urano y Neptuno}

\begin{marginfigure}
  \includegraphics[width=\linewidth]{img/solar_system_temperatures}
  \caption{Temperaturas de los planetas del Sistema Solar.}
\end{marginfigure}

Tras Júpiter y Saturno se sitúan los dos planetas helados e invisibles al ojo humano en condiciones normales desde la Tierra: Urano y Neptuno. Son los dos planetas más desconocidos y la única sonda de la que tenemos datos es la Voyager 2 a finales de 1980.

Ambos planetas poseen tamaño (aproximadamente $50.000~[km]$), masa (alrededor de $14.5$ y $17$ veces la masa de la Tierra respectivamente) y composición similares. Los dos tienen un tenue sistema de anillos y gran cantidad de satélites.

Las atmósferas congeladas del exterior de ambos planetas contienen cantidades similares de los componentes principales hidrógeno, helio y metano (hielos en suspensión, lo que les proporciona su famoso color azulado). Los mantos de ambos planetas están compuestos de agua, amoníaco y metano.

Urano no genera mucho calor interno por su cuenta por lo que se mantiene helado a una temperatura aproximada de $-220$ grados centígrados. En general, no tiene un clima cambiante dado que su eje de rotación hace que un polo esté constantemente (con un período de 42 años) hacia el sol.

Neptuno por su parte posee el clima más extremo, con vientos de hasta $2.100~[km/h]$ y tormentas de meses e incluso años. Su temperatura media es similar a la de Urano aunque su núcleo sí que emite una gran cantidad de calor.

\subsection{El Cinturón de Kuiper}

Es un vasto anillo de cuerpos rocosos de diversos tamaños que orbitan más allá de la órbita de Neptuno. Son los restos de la formación del Sistema Solar.

Postulada su existencia por Kuiper y Edgeworth para explicar la presencia de cometas con períodos orbitales superiores al siglo. Se estima que puede contener más de $100,000$ objetos más grandes de $100~[km]$.

Los dos cuerpos más notables son los planetas enanos Haumea y Makemake.

\subsection{Plutón y Caronte}

Plutón es un sistema doble de cuerpos del cinturón de Kuiper capturado por Neptuno que se consideró durante un tiempo el noveno planeta.

Es más pequeño que la Luna ($2400~[km]$ y $0.002~M_{\bigoplus}$) y posee una luna, Caronte, de tamaño comparable.

Tiene una órbita excéntrica y fuera del plano que de hecho es a veces más cercana al Sol que Neptuno. Fue estudiado por la New Horizons en 2005.

\subsection{Objetos Transneptunianos (TNOs)}

No todos los objetos situados más allá de Neptuno son parte del cinturón de Kuiper. Algunos de ellos simplemente tienen órbitas elípticas muy inclinadas.

Por ejemplo, Eris es el planeta enano más masivo del Sistema Solar. Con su descubrimiento en 2005 se creó el concepto de planeta enano y se degradó a Plutón.

La diferencia entre un planeta y un planeta enano (ambos orbitan al Sol y tienen masa suficiente como para que su propia gravedad les de forma esférica) es que los planetas enanos no han limpiado el entorno de su órbita de objetos menores. Los planetas enanos además no son satélites.

Los asteroides y cometas son formaciones rocosas, de hielo y polvo. La diferencia entre cometas y asteroides es bastante difusa. Al acercarse al sol desarrollan una estructura en forma de coma en torno al núcleo y colas (plasma empujado por el viento solar y la radiación) que apuntan en la dirección contraria al Sol.

\subsection{La Nube de Oort}

En 1950, Jan Oort propuso la existencia de una nube inmensa de proto-cometas orbitando el Sol. Cuando un efecto gravitatorio (una estrella que pasa cerca) perturba la nube, se liberan algunos de sus cuerpos hacia el Sol: cometas de muy largo período.

Contiene entre billones y hasta dos trillones de objetos helados principalmente compuestos de amoníaco, agua y metano. Se estima que tiene una anchura de un año luz y comienza a partir de $2000~[UA]$ de la Tierra.

La nube no ha sido detectada observacionalmente y es difícil que lo sea.

\subsection{El Planeta 9}

En 2016 se propuso la existencia de otro planeta más allá del cinturón de Kuiper, causante de las extrañas formas de las órbitas de algunos TNOs. Tendría una órbita muy amplia y excéntrica y una masa comparable a la de Urano o Neptuno. No hay señal de su existencia.

\subsection{Formación del Sistema Solar}

Uno de los hechos más destacables es que todos los planetas se encuentran en prácticamente el mismo plano y orbitan en la misma dirección alrededor del Sol. El Sol también gira en la misma dirección. Este patrón se interpreta como que el Sol y los planetas se formaron juntos por una nube de gas y polvo llamada la nebulosa solar (conservación del momento angular).

La composición de los planetas nos da otra pista sobre su origen: el Sol tiene la misma composición dominada por hidrógeno que Júpiter y Saturno y por lo tanto aparentemente fue formado por el mismo cúmulo de material. En comparativa, los planetas terrestres y la Luna son relativamente deficientes en gases ligeros, en ellos encontramos elementos más pesados como hierro y silicio. Esto sugiere que el proceso que llevó a la formaciónd el os planetas en el Sistema Solar interior por algún motivo excluyó estos materiales ligeros que escaparon dejando un residuo de materiales pesados.

La explicación a este fenómeno se debe principalmente al calor del Sol. Los planetas interiores y la mayoría de los asteroides están compuestos de rocas y metales capaces de sobrevivir a temperaturas elevadas.

La respuesta la podemos encontrar también en el exterior, observando la formación de planetas en estrellas lejanas más jóvenes. Hemos observado que existen muchas otras nebulosas solares y discos circumstelares (nubes aplanadas de gas y polvo girando y rodeando a estrellas jóvenes) que se parecerían a las primeras etapas de nuestro Sistema Solar.

Estos discos son muy frecuentes en estrellas jóvenes lo cual sugiere que se forman juntos. En ellos, la materia comienza a condensarse formando en primer lugar precursores de planetas llamados planetesimales. Estos planetesimales se juntarían debido a su gravedad mutua para dar lugar a planetas normales. Estos planetesimales chocarían violentamente y como consecuencia produciendo un calentamiento hasta que se convirtieran en líquido y gas y por lo tanto se diferenciaran unos de otros. Lo cual explica las diferentes composiciones de los planetas (según el gradiente de temperatura del disco primordial y los materiales que pueden condensarse).

\subsection{Exoplanetas}

Se trata de mundos (planetas) que existen fuera de nuestro Sistema Solar. Estos cuerpos suelen orbitar su propia estrella, de la misma forma que la Tierra lo hace alrededor del Sol, y suelen formar parte de sus propios sistemas solares.

Estos planetas están compuestos por los mismos materiales que la Tierra y los planetas de nuestro Sistema Solar.

Existen cuatro tipos: tipo gigante o parecidos a Neptuno (gigantes gaseosos), Júpiter caliente (gigantes gaseosos muy calientes cerca de su propia estrella), súper-Tierras (principalmente compuestos de rocas o hielos, más grandes que la Tierra pero menores que Júpiter) y análogos a la Tierra (similar tamaño, composición y distancia a su estrella).

Los exoplanetas son difíciles de detectar directamente, principalmente porque su brillo palidece en comparación con el de la estrella que orbitan.

Existen cinco métodos principales para detectarlos: velocidad radial (el planeta hace que la estrella se "tambalee"), imagen directa (bloqueando el brillo de la estrella para poder observar objetos cercanos), astrometría (observando el movimiento de la estrella en relación a otras estrellas cercanas), \emph{gravitational microlensing} (observando cambios de la trayectoria de la luz en presencia del campo gravitatorio de dichos planetas) y tránsitos (cuando el planeta bloquea la luz de la estrella) la más común.

La primera confirmación de un exoplaneta se produjo en 1995, 51 Pegasi B. Desde entonces se han encontrado más de 4000 planetas siendo los menos comunes los análogos a la Tierra.

Algunos de los más notables son Kepler-10b, Trappist-1C, Trappist-1E, Trappist-1F.

Según la teoría, por lo menos un exoplaneta orbita cada estrella en la Vía Láctea, lo cual sitúa la cuenta alrededor de un trillón.

\clearpage

\section{Sistemas Planetarios}

\clearpage

\section{Medio Interestelar}

\subsection{Polvo Cósmico}

\paragraph{Detección de Polvo}

Las nubes de polvo interestelar están demasiado frías como para irradiar energía en el espectro visible pero sí brillan con intensidad en el infrarrojo. Los pequeños granos de polvo absorben luz visible y radiación ultravioleta con eficiencia por lo que se calientan a temperaturas de $10-500~[K]$ radiando este calor en el infrarrojo.

\paragraph{Nebulosas de Reflexión}

Las \emph{Reflection nebula} ocurren cuando las nubes de polvo estelar se encuentran tan cerca de estrellas luminosas y hacen \emph{scatter} de suficiente luz como para que sean visibles. Lo que vemos, es la luz de la estrella reflejada en los granos de polvo estelar. Frecuentemente, parecen más azuladas que la luz real de la estrella que las ilumina ya que esas pequeñas partículas reflejan la luz azul con mayor eficiencia que la roja.

\paragraph{Nebulosas de Emisión}

TODO

\paragraph{Extinción Interestelar}

Los pequeños granos de polvo interestelar absorben parte de la luz de las estrellas que interceptan. Pero al menos la mitad de dicha luz es dispersada en lugar de absorberla. Como ni la luz absorbida ni la reflejada nos llega directamente, las estrellas parecen más tenues. El efecto combinado de estos dos procesos se denomina extinción.

Cabe destacar que el polvo no interactúa con todos los colores de la luz de la misma forma. La mayor parte de la luz azul, violeta y verde es dispersada y no nos alcanza. Mientras que otras luces con mayor longitud de onda como naranjas y rojos penetran en el polvo y sí que nos llegan a nosotros. Por este fenómeno, las estrellas parecen más rojas aunque sus líneas espectrales indiquen lo contrario.

\begin{figure}
  \includegraphics[width=\linewidth]{img/scatter}
  \caption{Dispersión o \emph{scattering} de la luz producida por polvo estelar.}
\end{figure}

\paragraph{Granos Interestelares}

La absorción y por lo tanto la extinción interestelar no puede provenir del gas. Tomando por ejemplo la Tierra, incluso con la altísima densidad de gas que posee nuestra atmósfera comparada con el medio interestelar, el gas sigue siendo prácticamente invisible por lo que la cantidad de gas interestelar debería mucho mayor de la existente para producir la absorción. En definitiva, el causante de este fenómeno debe ser un conjunto de partículas muy dispersas y de diversa composición: principalmente hidrógeno, helio, oxígeno, carbono y nitrógeno.

Estas partículas en su mayoría pueden caracterizarse como \emph{sootlike} (ricas en carbón) o \emph{sandlike} (conteniendo silicio y oxígeno). El modelo más extendido de estos granos supone un núcleo rocoso rico en carbono o silicatos rodeado de mantos helados de agua, metano o amoníaco.

Estos granos deben ser ligeramente inferiores en tamaño a la longitud de onda de la luz visible en torno a $10-100~[nm]$(si fueran más pequeños, la luz no sería bostruida; si fueran mayores, no se produciría el enrojecimiento).

\subsection{Rayos Cósmicos}

Además de gas y polvo, existe un tercer tipo de partículas que se encuentra en el espacio interestelar: rayos cósmicos. Se trata de partículas de composición similar al gas interestelar pero que viajan a altas velocidades (típicamente al $90~\%$ de la velocidad de la luz). La mayoría de ellos son núcleos de hidrógeno sin su electrón.

Dado que se tratan de partículas con carga, su trayectoria se ve afectada por campos magnéticos; eso provoca que sea casi imposible determinar su origen. Se supone que se originan en el interior de la galaxia (puesto que los campos magnéticos en el espacio interestelar son suficientemente fuertes como para prevenir su escape); el mejor candidato para su origen es la explosión de una supernova.

\subsection{El Ciclo Vital del Material Cósmico}

\paragraph{Flujo de Gas Interestelar}

La masa total del medio interestelar depende del tira y afloja entre aquella que se gana (por gravedad) del espacio intergaláctico, la conversión de parte de la misma en estrellas, y la pérdida provocada por masa devuelta de nuevo al espacio intergaláctico por la explosión de supernovas. Este proceso se conoce como el ciclo de \emph{baryon}.

\paragraph{Ciclo del Polvo y Elementos Pesados}

Aunque la mayor parte de la masa del medio interestelar se debe a material acretado del espacio intergaláctico, eso no explica los elementos más pesados que el hidrógeno y el helio ni tampoco el polvo estelar.

Los elementos pesados son formados en el interior de las estrellas de la Vía Láctea y son devueltos al medio interestelar al final de sus vidas.

Un proceso similar ocurre para los granos de polvo cósmico, que se forman al condensarse en regiones donde el gas es denso y frío (por ejemplo en los vientos de estrellas luminosas frías) o al enfriarse los gases eyectados por una supernova. Los "escudos" de hielo de estos granos serán evaporados por estrellas calientes e incorporados a su masa, creando de nuevo un ciclo.

\subsection{Material Interestelar alrededor del Sol}

\begin{figure}
  \includegraphics[width=\linewidth]{img/xray}
  \caption{Cielo en rayos X capturado por el satélite ROSAT; en rojo las regiones menos energéticas y en azul las más.}
\end{figure}

Los telescopios de rayos X han mostrado que la Galaxia está llena de burbujas de gases calientes emisores de rayos X. También han observado un fondo difuso de rayos X en todo el cielo desde nuestra perspectiva. Aunque una parte de ellos proviene de la interacción del viento solar con el medio interestelar, gran parte proviene de fuera del sistema solar. La explicación lógica a que haya gas caliente emisor de rayos X alrededor nuestro es que el Sol esté localizado en una de dichas burbujas conocida como la \emph{Local Hot Bubble}, mucho menos densa que el la densidad media interestelar (1 átomo por $cm^3$).

\subsection{Transporte Radiativo}

La cantidad de energía $dI_\nu$ que atraviesa un medio con absorción $\alpha_\nu$ y emisividad $j_\nu$ en una distancia $ds$ se define como:

\begin{align}
\frac{dI_\nu}{ds} = j_\nu - \alpha_\nu I_\nu~,
\end{align}

donde $I_\nu$ es la intensidad específica. En el caso sin emisión ($j_\nu = 0$):

\begin{align}
  \frac{dI_\nu}{ds} = - \alpha_\nu I_\nu
\end{align}

Si suponemos que el coeficiente de absorción es constante a lo largo de la trayectoria entonces:

\begin{align}
  \frac{dI_\nu(s)}{ds} = - \alpha_\nu I_\nu(s)\\
  I_\nu = I_{\nu_0} e^{- \alpha_\nu s}
\end{align}

Esto quiere decir que si la radiación de cierta intensidad específica atraviesa una nube de absorción, se produce un decaimiento exponencial en la intensidad de dicha radiación.

Podemos expresar $\tau_\nu \equiv \int \alpha_\nu(s) ds$, esto se conoce como la profundidad óptica u \emph{optical depth}.

\begin{align}
  I_\nu = I_{\nu_0} e^{-\tau_\nu}
\end{align}

Si $\tau_\nu \gg 1$ el medio se conoce como \emph{optically thick} (es muy difícil ver a través de un medio de dichas características). Por el contrario, si $\tau_\nu \ll 1$ el medio es \emph{optically thin}.

Podemos por lo tanto expresar la ecuación de transporte radiativo en función de la profundidad óptica:

\begin{align}
  \frac{dI_\nu}{ds}\frac{ds}{d\tau_\nu} = [j_\nu - \alpha_\nu I_\nu]\frac{ds}{d\tau_\nu}\\
  \frac{dI_\nu}{d\tau_\nu} = \frac{j_\nu}{\alpha_\nu} - I_\nu
\end{align}

De esta forma, aparece un ratio entre la emisión y la absorción conocido como la \emph{source function} $ S_\nu = j_\nu / \alpha_\nu$.

Así pues, podemos expresar la solución formal de la ecuación de transporte radiativo como:

\begin{align}
I_\nu(\tau_\nu) = I_{\nu_0}e^{-\tau_\nu} + \int_0^{\tau_\nu} S_\nu(\tau_\nu')e^{-(\tau_\nu - \tau_\nu')}d\tau_\nu'
\end{align}

En ella, el primer término se conoce como \emph{background light} mientras que la integral es el \emph{glowing medium}; la exponecial recibe el nombre de \emph{self absorption}.

Si asumimos que la \emph{source function} es constante (ni la emisión ni la absorción dependen de la profundidad óptica), la solución se simplifica notablemente:

\begin{align}
I_\nu(\tau_\nu) = I_{\nu_0}e^{-\tau_\nu} + S_\nu (1-e^{-\tau_\nu})
\end{align}

\section{Medio Interestelar: Review Questions}

\paragraph{Identify several dark nebulae in photographs in this chapter. Give the figure numbers of the photographs,
and specify where the dark nebulae are to be found on them.}

\paragraph{\textbf{Why do nebulae near hot stars look red? Why do dust clouds near stars usually look blue?}}

El gas interestelar cerca de las estrellas calientes es a su vez calentado a temperaturas de hasta $10000~[K]$, a su vez, la radiación ultravioleta de la estrella ioniza las nubes de gas. Como el hidrógeno es el principal componente de las mismas, se convierten en regiones $HII$ que están continuamente capturando electrones libres y perdiéndolos por ionización. Durante este proceso, emiten luz por fluorescencia y como generalmente el hidrógeno es su principal constituyente, la línea más fuerte en el espectro visible es la roja del hidrógeno. Este fenómeno se conoce como nebulosas de emisión.

Por otro lado, las nubes de polvo reflejan y dispersan la luz emitida por las estrellas cercanas. Al impactar contra el polvo estelar, la luz de longitud de onda azul se refleja de forma más eficiente que la roja por lo que parecen emitir un brillo azul. Se conocen como nebulosas de reflexión.

\paragraph{\textbf{Describe the characteristics of the various kinds of interstellar gas (HII regions, neutral hydrogen clouds,
ultra-hot gas clouds, and molecular clouds).}}

Las regiones $HII$ se tratan de gases (principalmente compuestos de hidrógeno) cerca de estrellas calientes que son ionizados por la radiación ultravioleta de dichas estrellas. Emiten luz roja por fluorescencia y se conocen como nebulosas de emisión.

Las nubes de hidrógeno neutral son las más comunes ya que las estrellas calientes necesarias para las regiones $HII$ son escasas y solamente una pequeña fracción del gas interestelar se encuentra lo suficientemente cerca para que se produzcan. A temperaturas típicas del medio interestelar ni absorben ni emiten luz en el espectro visible. Son detectables por líneas de absorción estrechas en determinadas partes fuera del espectro visible (principalmente a causa de calcio y sodio).

Las nubes de gas ultra-calientes se encuentran a temperaturas de un millón de grados centígrados o más (descubierto debido a que contienen átomos de oxígeno ionizados cinco veces). Se teoriza que este tipo de nubes son producidas por la explosión de una supernova.

Las nubes moleculares son regiones en las que la interacción gravitatoria ha atraído gas interestelar los suficiente para formar estructuras masivas de moléculas procedentes de la acumulación de polvo estelar en dicha región por la atracción gravitatoria. Estas nubes masivas producen una nube oscura que bloquean la luz ultravioleta y visible de las estrellas.

\paragraph{\textbf{Prepare a table listing the different ways in which dust and gas can be detected in interstellar space.}}

(1) Bloqueando directamente la luz visible, (2) reflejando/absorbiendo la luz de una estrella cercana, (3) emisiones de infrarrojos.

\paragraph{\textbf{Describe how the 21-cm line of hydrogen is formed. Why is this line such an important tool for
understanding the interstellar medium?}}

Un átomo de hidrógeno neutro excitado (en nuestro caso por la colisión con otros átomos de hidrógeno o con electrones libres), puede acabar perdiendo ese exceso de energía de excitación de nuevo o bien colisionando con otra partícula o bien emitiendo radiación con una longitud de onda de $21~[cm]$. Este proceso es lento (de media se produce una espera de $10~[Myr]$); aunque, dada la inmensa cantidad de átomos de hidrógeno en una nube neutra se produce con bastante asiduidad. La creación de equipo con sensibilidad suficiente para dicha longitud de onda ha permitido la detección de nubes de hidrógeno neutro en el medio interestelar.

\paragraph{\textbf{Describe the properties of the dust grains found in the space between stars.}}

TODO

\paragraph{\textbf{Why is it difficult to determine where cosmic rays come from?}}

Las partículas de un rayo cósmico están cargadas por lo que se ven afectadas por los diversos campos magnéticos presentes en el medio interestelar que desvían su trayectoria. Además, por la propia acción del campo magnético terrestre suelen sufrir varios cambios en la trayectoria antes de alcanzar la atmósfera en la cual los podemos detectar.

\paragraph{\textbf{What causes reddening of starlight? Explain how the reddish color of the Sun’s disk at sunset is caused by
the same process.}}

Los granos de polvo interestelar producen dispersión de las longitudes de onda azules mientras que las rojas apenas se ven afectadas; esto provoca que cuando dicha radiación nos llega a nosotros como observadores capturamos más luz roja que la que realmente emiten las estrellas. En realidad, el proceso es más bien un \emph{deblueing} más que \emph{reddening}. Este mismo proceso pero producido por las partículas presentes en nuestra atmósfera causa que el cielo tenga un color azulado (\emph{scattering}) mientras que el disco del sol parezca más rojo en la puesta (dado también el ángulo de incidencia de los rayos).

\paragraph{\textbf{Why do molecules, including $H_2$ and more complex organic molecules, only form inside dark clouds? Why
don’t they fill all interstellar space?}}

Únicamente se forman en el interior de nubes oscuras debido a que esa acumulación de materia bloquea la radiación ultravioleta de las estrellas cercanas, impidiendo así el proceso de ionización que a su vez haría imposible la formación de moléculas mayores y complejas. Esto no ocurre en todo el espacio interestelar puesto que los granos de polvo son una mínima parte de su composición y estas nubes masivas únicamente se forman en regiones muy concretas debido a atracción gravitatoria significativa.

\paragraph{\textbf{Why can't we use visible light telescopes to study molecular clouds where stars and planets form? Why do
infrared or radio telescopes work better?}}

Las nubes moleculares emiten radiación en una longitud de onda mayor que la de la luz visible. Es por ello que los telescopios infrrarojos ofrecen una mejor solución para captar dicha radiación.

\paragraph{The mass of the interstellar medium is determined by a balance between sources (which add mass) and
sinks (which remove it). Make a table listing the major sources and sinks, and briefly explain each one.}

\paragraph{Where does interstellar dust come from? How does it form?}

\section{Medio Interestelar: Thought Questions}

\clearpage

\section{Galaxias}

\clearpage

\section{Cosmología Física}

\clearpage

% \subsection{Sidenotes}\label{sec:sidenotes}
% One of the most prominent and distinctive features of this style is the
% extensive use of sidenotes.  There is a wide margin to provide ample room
% for sidenotes and small figures.  Any \Verb|\footnote|s will automatically
% be converted to sidenotes.\footnote{This is a sidenote that was entered
% using the \texttt{\textbackslash footnote} command.}  If you'd like to place ancillary
% information in the margin without the sidenote mark (the superscript
% number), you can use the \Verb|\marginnote| command.\marginnote{This is a
% margin note.  Notice that there isn't a number preceding the note, and
% there is no number in the main text where this note was written.}

% The specification of the \Verb|\sidenote| command is:
% \begin{docspec}
%   \doccmd{sidenote[\docopt{number}][\docopt{offset}]\{\docarg{Sidenote text.}\}}
% \end{docspec}

% Both the \docopt{number} and \docopt{offset} arguments are optional.  If you
% provide a \docopt{number} argument, then that number will be used as the
% sidenote number.  It will change of the number of the current sidenote only and
% will not affect the numbering sequence of subsequent sidenotes.

% Sometimes a sidenote may run over the top of other text or graphics in the
% margin space.  If this happens, you can adjust the vertical position of the
% sidenote by providing a dimension in the \docopt{offset} argument.  Some
% examples of valid dimensions are:
% \begin{docspec}
%   \ttfamily 1.0in \qquad 2.54cm \qquad 254mm \qquad 6\Verb|\baselineskip|
% \end{docspec}
% If the dimension is positive it will push the sidenote down the page; if the
% dimension is negative, it will move the sidenote up the page.

% While both the \docopt{number} and \docopt{offset} arguments are optional, they
% must be provided in order.  To adjust the vertical position of the sidenote
% while leaving the sidenote number alone, use the following syntax:
% \begin{docspec}
%   \doccmd{sidenote[][\docopt{offset}]\{\docarg{Sidenote text.}\}}
% \end{docspec}
% The empty brackets tell the \Verb|\sidenote| command to use the default
% sidenote number.

% If you \emph{only} want to change the sidenote number, however, you may
% completely omit the \docopt{offset} argument:
% \begin{docspec}
%   \doccmd{sidenote[\docopt{number}]\{\docarg{Sidenote text.}\}}
% \end{docspec}

% The \Verb|\marginnote| command has a similar \docarg{offset} argument:
% \begin{docspec}
%   \doccmd{marginnote[\docopt{offset}]\{\docarg{Margin note text.}\}}
% \end{docspec}

% \subsection{References}
% References are placed alongside their citations as sidenotes,
% as well.  This can be accomplished using the normal \Verb|\cite|
% command.\sidenote{The first paragraph of this document includes a citation.}

% The complete list of references may also be printed automatically by using
% the \Verb|\bibliography| command.  (See the end of this document for an
% example.)  If you do not want to print a bibliography at the end of your
% document, use the \Verb|\nobibliography| command in its place.  

% To enter multiple citations at one location,\cite{Tufte2006,Tufte1990} you can
% provide a list of keys separated by commas and the same optional vertical
% offset argument: \Verb|\cite{Tufte2006,Tufte1990}|.  
% \begin{docspec}
%   \doccmd{cite[\docopt{offset}]\{\docarg{bibkey1,bibkey2,\ldots}\}}
% \end{docspec}

% \section{Figures and Tables}\label{sec:figures-and-tables}
% Images and graphics play an integral role in Tufte's work.
% In addition to the standard \docenv{figure} and \docenv{tabular} environments,
% this style provides special figure and table environments for full-width
% floats.

% Full page--width figures and tables may be placed in \docenv{figure*} or
% \docenv{table*} environments.  To place figures or tables in the margin,
% use the \docenv{marginfigure} or \docenv{margintable} environments as follows
% (see figure~\ref{fig:marginfig}):

% \begin{marginfigure}%
%   \includegraphics[width=\linewidth]{helix}
%   \caption{This is a margin figure.  The helix is defined by 
%     $x = \cos(2\pi z)$, $y = \sin(2\pi z)$, and $z = [0, 2.7]$.  The figure was
%     drawn using Asymptote (\url{http://asymptote.sf.net/}).}
%   \label{fig:marginfig}
% \end{marginfigure}
% \begin{Verbatim}
% \begin{marginfigure}
%   \includegraphics{helix}
%   \caption{This is a margin figure.}
% \end{marginfigure}
% \end{Verbatim}

% The \docenv{marginfigure} and \docenv{margintable} environments accept an optional parameter \docopt{offset} that adjusts the vertical position of the figure or table.  See the ``\nameref{sec:sidenotes}'' section above for examples.  The specifications are:
% \begin{docspec}
%   \doccmd{begin\{marginfigure\}[\docopt{offset}]}\\
%   \qquad\ldots\\
%   \doccmd{end\{marginfigure\}}\\
%   \mbox{}\\
%   \doccmd{begin\{margintable\}[\docopt{offset}]}\\
%   \qquad\ldots\\
%   \doccmd{end\{margintable\}}\\
% \end{docspec}

% Figure~\ref{fig:fullfig} is an example of the \Verb|figure*|
% environment and figure~\ref{fig:textfig} is an example of the normal
% \Verb|figure| environment.

% \begin{figure*}[h]
%   \includegraphics[width=\linewidth]{sine.pdf}%
%   \caption{This graph shows $y = \sin x$ from about $x = [-10, 10]$.
%   \emph{Notice that this figure takes up the full page width.}}%
%   \label{fig:fullfig}%
% \end{figure*}

% \begin{figure}
%   \includegraphics{hilbertcurves.pdf}
% %  \checkparity This is an \pageparity\ page.%
%   \caption{Hilbert curves of various degrees $n$.
%   \emph{Notice that this figure only takes up the main textblock width.}}
%   \label{fig:textfig}
%   %\zsavepos{pos:textfig}
%   \setfloatalignment{b}
% \end{figure}

% Table~\ref{tab:normaltab} shows table created with the \docpkg{booktabs}
% package.  Notice the lack of vertical rules---they serve only to clutter
% the table's data.

% \begin{table}[ht]
%   \centering
%   \fontfamily{ppl}\selectfont
%   \begin{tabular}{ll}
%     \toprule
%     Margin & Length \\
%     \midrule
%     Paper width & \unit[8\nicefrac{1}{2}]{inches} \\
%     Paper height & \unit[11]{inches} \\
%     Textblock width & \unit[6\nicefrac{1}{2}]{inches} \\
%     Textblock/sidenote gutter & \unit[\nicefrac{3}{8}]{inches} \\
%     Sidenote width & \unit[2]{inches} \\
%     \bottomrule
%   \end{tabular}
%   \caption{Here are the dimensions of the various margins used in the Tufte-handout class.}
%   \label{tab:normaltab}
%   %\zsavepos{pos:normaltab}
% \end{table}

% \section{Full-width text blocks}

% In addition to the new float types, there is a \docenv{fullwidth}
% environment that stretches across the main text block and the sidenotes
% area.

% \begin{Verbatim}
% \begin{fullwidth}
% Lorem ipsum dolor sit amet...
% \end{fullwidth}
% \end{Verbatim}

% \begin{fullwidth}
% \small\itshape\lipsum[1]
% \end{fullwidth}

% \section{Typography}\label{sec:typography}

% \subsection{Typefaces}\label{sec:typefaces}
% If the Palatino, \textsf{Helvetica}, and \texttt{Bera Mono} typefaces are installed, this style
% will use them automatically.  Otherwise, we'll fall back on the Computer Modern
% typefaces.

% \subsection{Letterspacing}\label{sec:letterspacing}
% This document class includes two new commands and some improvements on
% existing commands for letterspacing.

% When setting strings of \allcaps{ALL CAPS} or \smallcaps{small caps}, the
% letter\-spacing---that is, the spacing between the letters---should be
% increased slightly.\cite{Bringhurst2005}  The \Verb|\allcaps| command has proper letterspacing for
% strings of \allcaps{FULL CAPITAL LETTERS}, and the \Verb|\smallcaps| command
% has letterspacing for \smallcaps{small capital letters}.  These commands
% will also automatically convert the case of the text to upper- or
% lowercase, respectively.

% The \Verb|\textsc| command has also been redefined to include
% letterspacing.  The case of the \Verb|\textsc| argument is left as is,
% however.  This allows one to use both uppercase and lowercase letters:
% \textsc{The Initial Letters Of The Words In This Sentence Are Capitalized.}



% \section{Installation}\label{sec:installation}
% To install the Tufte-\LaTeX\ classes, simply drop the
% following files into the same directory as your \texttt{.tex}
% file:
% \begin{quote}
%   \ttfamily
%   tufte-common.def\\
%   tufte-handout.cls\\
%   tufte-book.cls
% \end{quote}

% % TODO add instructions for installing it globally



% \section{More Documentation}\label{sec:more-doc}
% For more documentation on the Tufte-\LaTeX{} document classes (including commands not
% mentioned in this handout), please see the sample book.

% \section{Support}\label{sec:support}

% The website for the Tufte-\LaTeX\ packages is located at
% \url{http://code.google.com/p/tufte-latex/}.  On our website, you'll find
% links to our \smallcaps{svn} repository, mailing lists, bug tracker, and documentation.

\bibliography{sample-handout}
\bibliographystyle{plainnat}



\end{document}