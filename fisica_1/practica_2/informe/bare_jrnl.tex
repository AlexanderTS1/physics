%% bare_jrnl.tex
%% V1.4b
%% 2015/08/26
%% by Michael Shell
%% see http://www.michaelshell.org/
%% for current contact information.
%%
%% This is a skeleton file demonstrating the use of IEEEtran.cls
%% (requires IEEEtran.cls version 1.8b or later) with an IEEE
%% journal paper.
%%
%% Support sites:
%% http://www.michaelshell.org/tex/ieeetran/
%% http://www.ctan.org/pkg/ieeetran
%% and
%% http://www.ieee.org/

%%*************************************************************************
%% Legal Notice:
%% This code is offered as-is without any warranty either expressed or
%% implied; without even the implied warranty of MERCHANTABILITY or
%% FITNESS FOR A PARTICULAR PURPOSE! 
%% User assumes all risk.
%% In no event shall the IEEE or any contributor to this code be liable for
%% any damages or losses, including, but not limited to, incidental,
%% consequential, or any other damages, resulting from the use or misuse
%% of any information contained here.
%%
%% All comments are the opinions of their respective authors and are not
%% necessarily endorsed by the IEEE.
%%
%% This work is distributed under the LaTeX Project Public License (LPPL)
%% ( http://www.latex-project.org/ ) version 1.3, and may be freely used,
%% distributed and modified. A copy of the LPPL, version 1.3, is included
%% in the base LaTeX documentation of all distributions of LaTeX released
%% 2003/12/01 or later.
%% Retain all contribution notices and credits.
%% ** Modified files should be clearly indicated as such, including  **
%% ** renaming them and changing author support contact information. **
%%*************************************************************************


% *** Authors should verify (and, if needed, correct) their LaTeX system  ***
% *** with the testflow diagnostic prior to trusting their LaTeX platform ***
% *** with production work. The IEEE's font choices and paper sizes can   ***
% *** trigger bugs that do not appear when using other class files.       ***                          ***
% The testflow support page is at:
% http://www.michaelshell.org/tex/testflow/



\documentclass[journal]{IEEEtran}
%
% If IEEEtran.cls has not been installed into the LaTeX system files,
% manually specify the path to it like:
% \documentclass[journal]{../sty/IEEEtran}





% Some very useful LaTeX packages include:
% (uncomment the ones you want to load)
\usepackage[spanish, es-tabla]{babel}
\usepackage[utf8]{inputenc}
\usepackage{url}
\usepackage{minted}
\setminted{fontsize=\small,baselinestretch=1}
\usepackage{listings}
\usepackage{subcaption}
\renewcommand\listingscaption{Código}
\newenvironment{code}{\captionsetup{type=listing}}{\par\addvspace{\baselineskip}}
\usepackage{graphicx}
\usepackage{tikz}
\usetikzlibrary{calc,patterns,angles,quotes}
\usetikzlibrary{babel}

% *** MISC UTILITY PACKAGES ***
%
%\usepackage{ifpdf}
% Heiko Oberdiek's ifpdf.sty is very useful if you need conditional
% compilation based on whether the output is pdf or dvi.
% usage:
% \ifpdf
%   % pdf code
% \else
%   % dvi code
% \fi
% The latest version of ifpdf.sty can be obtained from:
% http://www.ctan.org/pkg/ifpdf
% Also, note that IEEEtran.cls V1.7 and later provides a builtin
% \ifCLASSINFOpdf conditional that works the same way.
% When switching from latex to pdflatex and vice-versa, the compiler may
% have to be run twice to clear warning/error messages.






% *** CITATION PACKAGES ***
%
%\usepackage{cite}
% cite.sty was written by Donald Arseneau
% V1.6 and later of IEEEtran pre-defines the format of the cite.sty package
% \cite{} output to follow that of the IEEE. Loading the cite package will
% result in citation numbers being automatically sorted and properly
% "compressed/ranged". e.g., [1], [9], [2], [7], [5], [6] without using
% cite.sty will become [1], [2], [5]--[7], [9] using cite.sty. cite.sty's
% \cite will automatically add leading space, if needed. Use cite.sty's
% noadjust option (cite.sty V3.8 and later) if you want to turn this off
% such as if a citation ever needs to be enclosed in parenthesis.
% cite.sty is already installed on most LaTeX systems. Be sure and use
% version 5.0 (2009-03-20) and later if using hyperref.sty.
% The latest version can be obtained at:
% http://www.ctan.org/pkg/cite
% The documentation is contained in the cite.sty file itself.






% *** GRAPHICS RELATED PACKAGES ***
%
\ifCLASSINFOpdf
  % \usepackage[pdftex]{graphicx}
  % declare the path(s) where your graphic files are
  % \graphicspath{{../pdf/}{../jpeg/}}
  % and their extensions so you won't have to specify these with
  % every instance of \includegraphics
  % \DeclareGraphicsExtensions{.pdf,.jpeg,.png}
\else
  % or other class option (dvipsone, dvipdf, if not using dvips). graphicx
  % will default to the driver specified in the system graphics.cfg if no
  % driver is specified.
  % \usepackage[dvips]{graphicx}
  % declare the path(s) where your graphic files are
  % \graphicspath{{../eps/}}
  % and their extensions so you won't have to specify these with
  % every instance of \includegraphics
  % \DeclareGraphicsExtensions{.eps}
\fi
% graphicx was written by David Carlisle and Sebastian Rahtz. It is
% required if you want graphics, photos, etc. graphicx.sty is already
% installed on most LaTeX systems. The latest version and documentation
% can be obtained at: 
% http://www.ctan.org/pkg/graphicx
% Another good source of documentation is "Using Imported Graphics in
% LaTeX2e" by Keith Reckdahl which can be found at:
% http://www.ctan.org/pkg/epslatex
%
% latex, and pdflatex in dvi mode, support graphics in encapsulated
% postscript (.eps) format. pdflatex in pdf mode supports graphics
% in .pdf, .jpeg, .png and .mps (metapost) formats. Users should ensure
% that all non-photo figures use a vector format (.eps, .pdf, .mps) and
% not a bitmapped formats (.jpeg, .png). The IEEE frowns on bitmapped formats
% which can result in "jaggedy"/blurry rendering of lines and letters as
% well as large increases in file sizes.
%
% You can find documentation about the pdfTeX application at:
% http://www.tug.org/applications/pdftex





% *** MATH PACKAGES ***
%
%\usepackage{amsmath}
% A popular package from the American Mathematical Society that provides
% many useful and powerful commands for dealing with mathematics.
%
% Note that the amsmath package sets \interdisplaylinepenalty to 10000
% thus preventing page breaks from occurring within multiline equations. Use:
%\interdisplaylinepenalty=2500
% after loading amsmath to restore such page breaks as IEEEtran.cls normally
% does. amsmath.sty is already installed on most LaTeX systems. The latest
% version and documentation can be obtained at:
% http://www.ctan.org/pkg/amsmath





% *** SPECIALIZED LIST PACKAGES ***
%
%\usepackage{algorithmic}
% algorithmic.sty was written by Peter Williams and Rogerio Brito.
% This package provides an algorithmic environment fo describing algorithms.
% You can use the algorithmic environment in-text or within a figure
% environment to provide for a floating algorithm. Do NOT use the algorithm
% floating environment provided by algorithm.sty (by the same authors) or
% algorithm2e.sty (by Christophe Fiorio) as the IEEE does not use dedicated
% algorithm float types and packages that provide these will not provide
% correct IEEE style captions. The latest version and documentation of
% algorithmic.sty can be obtained at:
% http://www.ctan.org/pkg/algorithms
% Also of interest may be the (relatively newer and more customizable)
% algorithmicx.sty package by Szasz Janos:
% http://www.ctan.org/pkg/algorithmicx




% *** ALIGNMENT PACKAGES ***
%
%\usepackage{array}
% Frank Mittelbach's and David Carlisle's array.sty patches and improves
% the standard LaTeX2e array and tabular environments to provide better
% appearance and additional user controls. As the default LaTeX2e table
% generation code is lacking to the point of almost being broken with
% respect to the quality of the end results, all users are strongly
% advised to use an enhanced (at the very least that provided by array.sty)
% set of table tools. array.sty is already installed on most systems. The
% latest version and documentation can be obtained at:
% http://www.ctan.org/pkg/array


% IEEEtran contains the IEEEeqnarray family of commands that can be used to
% generate multiline equations as well as matrices, tables, etc., of high
% quality.




% *** SUBFIGURE PACKAGES ***
%\ifCLASSOPTIONcompsoc
%  \usepackage[caption=false,font=normalsize,labelfont=sf,textfont=sf]{subfig}
%\else
%  \usepackage[caption=false,font=footnotesize]{subfig}
%\fi
% subfig.sty, written by Steven Douglas Cochran, is the modern replacement
% for subfigure.sty, the latter of which is no longer maintained and is
% incompatible with some LaTeX packages including fixltx2e. However,
% subfig.sty requires and automatically loads Axel Sommerfeldt's caption.sty
% which will override IEEEtran.cls' handling of captions and this will result
% in non-IEEE style figure/table captions. To prevent this problem, be sure
% and invoke subfig.sty's "caption=false" package option (available since
% subfig.sty version 1.3, 2005/06/28) as this is will preserve IEEEtran.cls
% handling of captions.
% Note that the Computer Society format requires a larger sans serif font
% than the serif footnote size font used in traditional IEEE formatting
% and thus the need to invoke different subfig.sty package options depending
% on whether compsoc mode has been enabled.
%
% The latest version and documentation of subfig.sty can be obtained at:
% http://www.ctan.org/pkg/subfig




% *** FLOAT PACKAGES ***
%
%\usepackage{fixltx2e}
% fixltx2e, the successor to the earlier fix2col.sty, was written by
% Frank Mittelbach and David Carlisle. This package corrects a few problems
% in the LaTeX2e kernel, the most notable of which is that in current
% LaTeX2e releases, the ordering of single and double column floats is not
% guaranteed to be preserved. Thus, an unpatched LaTeX2e can allow a
% single column figure to be placed prior to an earlier double column
% figure.
% Be aware that LaTeX2e kernels dated 2015 and later have fixltx2e.sty's
% corrections already built into the system in which case a warning will
% be issued if an attempt is made to load fixltx2e.sty as it is no longer
% needed.
% The latest version and documentation can be found at:
% http://www.ctan.org/pkg/fixltx2e


%\usepackage{stfloats}
% stfloats.sty was written by Sigitas Tolusis. This package gives LaTeX2e
% the ability to do double column floats at the bottom of the page as well
% as the top. (e.g., "\begin{figure*}[!b]" is not normally possible in
% LaTeX2e). It also provides a command:
%\fnbelowfloat
% to enable the placement of footnotes below bottom floats (the standard
% LaTeX2e kernel puts them above bottom floats). This is an invasive package
% which rewrites many portions of the LaTeX2e float routines. It may not work
% with other packages that modify the LaTeX2e float routines. The latest
% version and documentation can be obtained at:
% http://www.ctan.org/pkg/stfloats
% Do not use the stfloats baselinefloat ability as the IEEE does not allow
% \baselineskip to stretch. Authors submitting work to the IEEE should note
% that the IEEE rarely uses double column equations and that authors should try
% to avoid such use. Do not be tempted to use the cuted.sty or midfloat.sty
% packages (also by Sigitas Tolusis) as the IEEE does not format its papers in
% such ways.
% Do not attempt to use stfloats with fixltx2e as they are incompatible.
% Instead, use Morten Hogholm'a dblfloatfix which combines the features
% of both fixltx2e and stfloats:
%
% \usepackage{dblfloatfix}
% The latest version can be found at:
% http://www.ctan.org/pkg/dblfloatfix




%\ifCLASSOPTIONcaptionsoff
%  \usepackage[nomarkers]{endfloat}
% \let\MYoriglatexcaption\caption
% \renewcommand{\caption}[2][\relax]{\MYoriglatexcaption[#2]{#2}}
%\fi
% endfloat.sty was written by James Darrell McCauley, Jeff Goldberg and 
% Axel Sommerfeldt. This package may be useful when used in conjunction with 
% IEEEtran.cls'  captionsoff option. Some IEEE journals/societies require that
% submissions have lists of figures/tables at the end of the paper and that
% figures/tables without any captions are placed on a page by themselves at
% the end of the document. If needed, the draftcls IEEEtran class option or
% \CLASSINPUTbaselinestretch interface can be used to increase the line
% spacing as well. Be sure and use the nomarkers option of endfloat to
% prevent endfloat from "marking" where the figures would have been placed
% in the text. The two hack lines of code above are a slight modification of
% that suggested by in the endfloat docs (section 8.4.1) to ensure that
% the full captions always appear in the list of figures/tables - even if
% the user used the short optional argument of \caption[]{}.
% IEEE papers do not typically make use of \caption[]'s optional argument,
% so this should not be an issue. A similar trick can be used to disable
% captions of packages such as subfig.sty that lack options to turn off
% the subcaptions:
% For subfig.sty:
% \let\MYorigsubfloat\subfloat
% \renewcommand{\subfloat}[2][\relax]{\MYorigsubfloat[]{#2}}
% However, the above trick will not work if both optional arguments of
% the \subfloat command are used. Furthermore, there needs to be a
% description of each subfigure *somewhere* and endfloat does not add
% subfigure captions to its list of figures. Thus, the best approach is to
% avoid the use of subfigure captions (many IEEE journals avoid them anyway)
% and instead reference/explain all the subfigures within the main caption.
% The latest version of endfloat.sty and its documentation can obtained at:
% http://www.ctan.org/pkg/endfloat
%
% The IEEEtran \ifCLASSOPTIONcaptionsoff conditional can also be used
% later in the document, say, to conditionally put the References on a 
% page by themselves.




% *** PDF, URL AND HYPERLINK PACKAGES ***
%
%\usepackage{url}
% url.sty was written by Donald Arseneau. It provides better support for
% handling and breaking URLs. url.sty is already installed on most LaTeX
% systems. The latest version and documentation can be obtained at:
% http://www.ctan.org/pkg/url
% Basically, \url{my_url_here}.




% *** Do not adjust lengths that control margins, column widths, etc. ***
% *** Do not use packages that alter fonts (such as pslatex).         ***
% There should be no need to do such things with IEEEtran.cls V1.6 and later.
% (Unless specifically asked to do so by the journal or conference you plan
% to submit to, of course. )


% correct bad hyphenation here
\hyphenation{op-tical net-works semi-conduc-tor}


\begin{document}
%
% paper title
% Titles are generally capitalized except for words such as a, an, and, as,
% at, but, by, for, in, nor, of, on, or, the, to and up, which are usually
% not capitalized unless they are the first or last word of the title.
% Linebreaks \\ can be used within to get better formatting as desired.
% Do not put math or special symbols in the title.
\title{Estudio de La Oscilación de un Péndulo}
%
%
% author names and IEEE memberships
% note positions of commas and nonbreaking spaces ( ~ ) LaTeX will not break
% a structure at a ~ so this keeps an author's name from being broken across
% two lines.
% use \thanks{} to gain access to the first footnote area
% a separate \thanks must be used for each paragraph as LaTeX2e's \thanks
% was not built to handle multiple paragraphs
%

\author{Alberto García García\\ (48718198-N)\\ \texttt{agg180@alu.ua.es} % <-this % stops a space
\thanks{}%
}

% note the % following the last \IEEEmembership and also \thanks - 
% these prevent an unwanted space from occurring between the last author name
% and the end of the author line. i.e., if you had this:
% 
% \author{....lastname \thanks{...} \thanks{...} }
%                     ^------------^------------^----Do not want these spaces!
%
% a space would be appended to the last name and could cause every name on that
% line to be shifted left slightly. This is one of those "LaTeX things". For
% instance, "\textbf{A} \textbf{B}" will typeset as "A B" not "AB". To get
% "AB" then you have to do: "\textbf{A}\textbf{B}"
% \thanks is no different in this regard, so shield the last } of each \thanks
% that ends a line with a % and do not let a space in before the next \thanks.
% Spaces after \IEEEmembership other than the last one are OK (and needed) as
% you are supposed to have spaces between the names. For what it is worth,
% this is a minor point as most people would not even notice if the said evil
% space somehow managed to creep in.



% The paper headers
\markboth{Física I -- Grado en Física -- 2018-2019}%
{}
% The only time the second header will appear is for the odd numbered pages
% after the title page when using the twoside option.
% 
% *** Note that you probably will NOT want to include the author's ***
% *** name in the headers of peer review papers.                   ***
% You can use \ifCLASSOPTIONpeerreview for conditional compilation here if
% you desire.




% If you want to put a publisher's ID mark on the page you can do it like
% this:
%\IEEEpubid{0000--0000/00\$00.00~\copyright~2015 IEEE}
% Remember, if you use this you must call \IEEEpubidadjcol in the second
% column for its text to clear the IEEEpubid mark.



% use for special paper notices
%\IEEEspecialpapernotice{(Invited Paper)}




% make the title area
\maketitle

% As a general rule, do not put math, special symbols or citations
% in the abstract or keywords.
\begin{abstract}
En esta segunda práctica de la asignatura Física I del Grado en Física (curso académico 2018-2019) estudiaremos la oscilación de un péndulo. Para ello consideraremos que el péndulo posee una masa y longitud preestablecidas y que se deja caer desde el reposo al apartarlo de su posición de equilibrio en cierta medida. En primer lugar, obtendremos las ecuaciones diferenciales que permiten calcular los diferentes componentes del péndulo. Seguidamente resolveremos
  dichas ecuaciones con un módulo de integración numérica para calcular la amplitud, la velocidad angular y la tensión de la cuerda. Con tal resolución, estudiaremos la dependencia entre el período de oscilación y la amplitud inicial. También compararemos los resultados numéricos obtenidos anteriormente con la solución analítica. Por último, introduciremos un término de rozamiento y estudiaremos su efecto sobre el péndulo y su período.

El código Python que implementa los modelos matemáticos así como las rutinas de visualización para la resolución de este ejercicio se adjunta con este informe y además puede ser consultado en el siguiente repositorio online \footnote{\url{https://github.com/Blitzman/physics}}.
\end{abstract}


% For peer review papers, you can put extra information on the cover
% page as needed:
% \ifCLASSOPTIONpeerreview
% \begin{center} \bfseries EDICS Category: 3-BBND \end{center}
% \fi
%
% For peerreview papers, this IEEEtran command inserts a page break and
% creates the second title. It will be ignored for other modes.
\IEEEpeerreviewmaketitle

\section{Introducción}

\IEEEPARstart{E}{l} péndulo es un cuerpo suspendido por una cuerda de un punto alrededor del cual oscila por acción de la fuerza gravitatoria. Podemos hablar de péndulo ideal cuando se asume que la cuerda no tiene masa y que el cuerpo carece de fricción o rozamiento con el medio. Es decir, que el péndulo oscilará indefinidamente.

En esta práctica consideraremos un péndulo de longitud $l$ y masa $m$ que se aparta de su posición de equilibrio un ángulo $\theta_0$ y se deja caer desde el reposo (tal y como se muestra en la Figura \ref{fig:pendulum}). Inicialmente supondremos un péndulo ideal para posteriormente introducir fuerzas amortiguadoras y estudiar el efecto de las mismas.

\begin{figure}[!hbt]
  \centering
\begin{tikzpicture}
    % save length of g-vector and theta to macros
    \pgfmathsetmacro{\Gvec}{1.5}
    \pgfmathsetmacro{\myAngle}{45}
    % calculate lengths of vector components
    \pgfmathsetmacro{\Gcos}{\Gvec*cos(\myAngle)}
    \pgfmathsetmacro{\Gsin}{\Gvec*sin(\myAngle)}

    \coordinate (centro) at (0,0);
    \draw[dashed,gray,-] (centro) -- ++ (0,-3.5) node (mary) [black,below]{$ $};
    \draw[thick] (centro) -- ++(270+\myAngle:3) coordinate (bob)
      node[midway, above right]{$l$};
    \pic [olive, draw, ->, "$\theta$", angle eccentricity=1.5] {angle = mary--centro--bob};
    \draw [blue,-stealth] (bob) -- ($(bob)!\Gcos cm!(centro)$)
      node[midway, above right]{$T$};
    \draw [-stealth] (bob) -- ($(bob)!-\Gcos cm!(centro)$)
      coordinate (gcos)
      node[midway,above right] {$mg\cos\theta$};
    \draw [-stealth] (bob) -- ($(bob)!\Gsin cm!90:(centro)$)
      coordinate (gsin)
      node[midway,above left] {$mg\sin\theta$};
    \draw [red,-stealth] ($(bob)!1.0*\Gsin cm!90:(centro)$) -- ($(bob)!2.0*\Gsin cm!90:(centro)$)
      coordinate (gsin)
      node[left] {$\vec{v}$};
    \draw [-stealth] (bob) -- ++(0,-\Gvec)
      coordinate (g)
      node[near end,left] {$g$};
    \pic [olive, draw, ->, "$\theta$", angle eccentricity=1.5] {angle = g--bob--gcos};
    \filldraw [fill=black!40,draw=black] (bob) circle[radius=0.1];
\end{tikzpicture}
  \caption{Diagrama de cuerpo libre de un péndulo de masa $m$ y cuerda de longitud $l$ apartado un ángulo $theta$ de la vertical.}
  \label{fig:pendulum}
\end{figure}

\section{Ecuaciones Diferenciales del Péndulo}

Las ecuaciones diferenciales que permiten calcular el ángulo $\theta$ y la tensión de la cuerda $T$ como función del tiempo pueden obtenerse a partir de las ecuaciones de Newton que rigen el funcionamiento del péndulo ideal.

El sumatorio de fuerzas sobre el eje radial ($y$) es

\begin{equation}
  \sum F_y = T - Pcos(\theta) = 0 \Rightarrow T = mgcos(\theta)~.
\end{equation}

Puesto que $\theta$ depende del tiempo (ya que la posición o ángulo del péndulo varía con el mismo), hemos encontrado la ecuación que permite calcular la tensión de la cuerda

\begin{equation}
  T(\theta(t)) = mgcos(\theta(t)) ~ [N]~.
\end{equation}

Por otro lado, en el eje tangencial o angular es

\begin{equation}
  \sum F_x = -Psin(\theta) = ma \Rightarrow -mgsin(\theta) = ma_\theta
\end{equation}
\begin{equation}
  \Rightarrow -gsin(\theta) = a_\theta
\end{equation}

Toda la aceleración producida en el eje tangencial podemos expresarla en coordenadas polares (teniendo en cuenta que el radio $l$ es constante) como

\begin{equation}
  a_\theta = 2\displaystyle\frac{dl}{dt}\displaystyle\frac{d\theta}{dt} + l\displaystyle\frac{d^2\theta}{dt^2} = l\displaystyle\frac{d^2\theta}{dt^2}~.
\end{equation}

Reemplazando la aceleración obtenida y reorganizando los términos teniendo en cuenta la aproximación de pequeñas oscilaciones ($sin(\theta) \sim \theta$) obtenemos

\begin{equation}
  \displaystyle\frac{d^2\theta}{dt^2} + \displaystyle\frac{g}{l}sin(\theta) = 0 ~.
\end{equation}
\begin{equation}
  \displaystyle\frac{d^2\theta}{dt^2} + \displaystyle\frac{g}{l}\theta = 0 ~.
\end{equation}

Esta ecuación describe un movimiento armónico simple cuya solución analítica para la posición o ángulo es

\begin{equation}
  \theta(t) = \theta_0 cos(\sqrt{\displaystyle\frac{g}{l}}t) = \theta_0 cos(\omega_0t) ~[rad]
\end{equation}

y para la velocidad angular

\begin{equation}
  \omega(t) = - \omega_0\theta_0 sin(\omega_0t) ~[rad/s]~.
\end{equation}

\section{Ángulo, Velocidad Angular y Período}

Una vez obtenidas las ecuaciones diferenciales que rigen el estado del péndulo en función del tiempo, podemos resolverlas para encontrar $\theta(t)$, $\omega(t)$ y $T(t)$ empleando la rutina \mintinline{python}{odeint} tal y como se muestra en el Código \ref{code:odeint}. A dicha rutina le proporcionaremos la función con las derivadas \mintinline{python}{pendulum} (mostrada en el Código \ref{code:derivs}), las condiciones iniciales
\mintinline{python}{z_0_}, los pasos de tiempo \mintinline{python}{t_values_} y los parámetros necesarios para la integración \mintinline{python}{args=([G, args.l],)} (la constante gravitatoria y la longitud del péndulo.

Posteriormente, del resultado \mintinline{python}{z_} extraeremos los valores de $\theta(t)$ en \mintinline{python}{z_[:,0]}, los de $\omega(t)$ en \mintinline{python}{z_[:,1]} y calcularemos los valores para la tensión empleando los valores anteriores.

\medskip

\begin{code}
  \begin{minted}{python}
w_0_ = 0.0
theta_0_ = np.radians(theta_0)
t_values_ = np.linspace(args.t0,
            args.tf,
            int((args.tf - args.t0) / args.dt))
z_0_ = [theta_0_, w_0_]
z_ = odeint(pendulum, z_0_, t_values_,
            args=([G, args.l], ))

thetas_ = z_[:,0]
omegas_ = z_[:,1]
tension_ = args.m * G * np.cos(z_[:,0])
  \end{minted}
  \caption{Parámetros de integración, llamada a rutina \mintinline{python}{odeint} y extracción de valores de $\theta(t)$, $\omega(t)$ y $T(t)$.}
  \label{code:odeint}
\end{code}

\begin{code}
  \begin{minted}{python}
def pendulum(y, t, params):
    phi, omega = y
    g, R = params
    derivs = [omega,
              -(g/R)*np.sin(phi)]
    return derivs
  \end{minted}
  \caption{Código de derivadas para \mintinline{python}{odeint}.}
  \label{code:derivs}
\end{code}

Los resultados para diferentes ángulos iniciales $\theta_0 = \{0.0, 5.0, 10.0, 20.0, 40.0\}~[rad]$, $l=5.0~[m]$ y $m=1.0~[kg]$ se muestran en las Figuras \ref{fig:theta}, \ref{fig:omega}, \ref{fig:tension} para $\theta(t)$, $\omega(t)$ y $T(t)$ respectivamente. En todos los casos $dt=0.001~[s]$, $t_0 = 0~[s]$ y $t_f = 10~[s]$.

\begin{figure}[!htb]
  \centering
  \includegraphics[width=\linewidth]{theta}
  \caption{Evolución del ángulo $\theta$ con el tiempo para diversos ángulos iniciales $\theta_0$.}
  \label{fig:theta}
\end{figure}

\begin{figure}[!htb]
  \centering
  \includegraphics[width=\linewidth]{omega}
  \caption{Evolución de la velocidad angular $\omega$ con el tiempo para diversos ángulos iniciales $\theta_0$.}
  \label{fig:omega}
\end{figure}

\begin{figure}[!htb]
  \centering
  \includegraphics[width=\linewidth]{tension}
  \caption{Evolución de la tensión $T$ con el tiempo para diversos ángulos iniciales $\theta_0$.}
  \label{fig:tension}
\end{figure}

\section{Estudio del Período con la Amplitud}

En este primer conjunto de experimentos trataremos de determinar cómo depende el período de la oscilación del péndulo $T$ con la amplitud inicial del mismo $\theta_0$. Para ello hemos solucionado las ecuaciones diferenciales expuestas anteriormente con la rutina \mintinline{python}{odeint} con el propósito de obtener la amplitud $\theta(t)$ en cada instante de tiempo y así poder comprobar la evolución de la misma.

Para estos experimentos hemos elegido un tiempo inicial $t_0 = 0~[s]$, un tiempo final $t = 10~[s]$, un intervalo de tiempo $dt = 0.01~[s]$, masa del péndulo $m = 1.0~[kg]$, longitud del mismo $l = 5.0~[m]$ y siete posibles valores para la amplitud inicial $\theta_0 = \{5, 10, 15, 30, 45, 60, 90\}~[deg]$.

\begin{figure}[!htb]
  \centering
  \includegraphics[width=\linewidth]{theta0}
  \caption{Evolución de la amplitud $\theta$ con el tiempo $t$ para diferentes amplitudes iniciales $\theta_0$. Valores calculados empleando la rutina de integración \mintinline{python}{odeint}.}
  \label{fig:theta0}
\end{figure}

La Figura \ref{fig:theta0} muestra las soluciones obtenidas a medida que incrementamos el valor del ángulo inicial $\theta_0$. Como podemos observar, para ángulos iniciales pequeños podemos realizar la aproximación $sin(\theta) \sim \theta$ y calcular su período como $T \sim 2\pi\sqrt{\frac{l}{g}}$. Para el caso que nos ocupa $T \sim 2\pi\sqrt{\frac{5}{9.8}} = 4.48~[s]$ y vemos que esta aproximación se cumple para los valores $\theta_0 = \{5, 10,
15\}$. A partir de $\theta_0 = 30$ se comienza a producir una desviación significativa respecto a ese período.

Así pues, podemos concluir con que existe una relación entre la amplitud inicial y el período del péndulo de manera que a medida que la amplitud aumenta también lo hace el propio período $T\propto \theta_0$.

\section{Comparación con Solución Analítica}

A continuación procederemos a comparar los resultados numéricos con las soluciones analíticas obtenidas empleando la aproximación de pequeñas oscilaciones implementada en el Código \ref{code:peqosc}. Las Figuras \ref{fig:numanal_5}, \ref{fig:numanal_10}, \ref{fig:numanal_20} y \ref{fig:numanal_40} muestran comparativas de los valores de $\theta$ para ambas soluciones con diferentes ángulos iniciales $\theta_0$. Podemos comprobar como la aproximación difiere de los resultados numéricos a medida que el tiempo transcurre, siendo esta diferencia acentuada conforme incrementamos el ángulo inicial $\theta_0$.

\medskip

\begin{code}
  \begin{minted}{python}
def small_angle_approx (theta0, w0, t):
  return theta0 * np.cos(w0 * t)
  \end{minted}
  \caption{Código de aproximación de pequeñas oscilaciones.}
  \label{code:peqosc}
\end{code}

\begin{figure}[!htb]
  \centering
  \includegraphics[width=\linewidth]{theta_5_diff}
  \caption{Evolución de la amplitud $\theta$ con el tiempo $t$ para amplitud inicial $\theta_0 = 5 [deg]$. Valores numéricos calculados empleando la rutina de integración \mintinline{python}{odeint} y analíticos empleando la aproximación de pequeñas oscilaciones.}
  \label{fig:numanal_5}
\end{figure}
\begin{figure}[!htb]
  \centering
  \includegraphics[width=\linewidth]{theta_10_diff}
  \caption{Evolución de la amplitud $\theta$ con el tiempo $t$ para amplitud inicial $\theta_0 = 10 [deg]$. Valores numéricos calculados empleando la rutina de integración \mintinline{python}{odeint} y analíticos empleando la aproximación de pequeñas oscilaciones.}
  \label{fig:numanal_10}
\end{figure}
\begin{figure}[!htb]
  \centering
  \includegraphics[width=\linewidth]{theta_20_diff}
  \caption{Evolución de la amplitud $\theta$ con el tiempo $t$ para amplitud inicial $\theta_0 = 20 [deg]$. Valores numéricos calculados empleando la rutina de integración \mintinline{python}{odeint} y analíticos empleando la aproximación de pequeñas oscilaciones.}
  \label{fig:numanal_20}
\end{figure}
\begin{figure}[!htb]
  \centering
  \includegraphics[width=\linewidth]{theta_40_diff}
  \caption{Evolución de la amplitud $\theta$ con el tiempo $t$ para amplitud inicial $\theta_0 = 40 [deg]$. Valores numéricos calculados empleando la rutina de integración \mintinline{python}{odeint} y analíticos empleando la aproximación de pequeñas oscilaciones.}
  \label{fig:numanal_40}
\end{figure}

La Tabla \ref{table:tiempodiff} muestra el tiempo que transcurre para cada uno de dichos experimentos mostrados en las gráficas hasta que el resultado analítico $\theta_T(t)$ y el numérico $\theta(t)$ difieren en más de un umbral determinado ($1\%$), es decir $\left|\displaystyle\frac{\theta(t) - \theta_T(t)}{\theta_T(t)}\right|>0.01$.

Esta diferencia se puede apreciar también en otra vista proporcionada por la Figura \ref{fig:diff}, la cual muestra la diferencia angular $\theta$ entre ambas soluciones a medida que transcurre el tiempo.

\begin{table}[!htb]
  \centering
  \begin{tabular}{c|cccc}
    & $\theta_0 = 5$ & $\theta_0 = 10$ & $\theta_0 = 20$ & $\theta_0 = 40$\\
    \hline
    Tiempo $[s]$ & $1.08$ & $0.94$ & $0.64$ & $0.35$\\
    \hline
  \end{tabular}
  \caption{Tiempo transcurrido hasta que la solución analítica y la numérica difieren en más de un $1\%$ para diferentes valores de ángulo inicial $\theta_0 = \{5, 10, 20, 40\} [deg]$.}
  \label{table:tiempodiff}
\end{table}

\begin{figure}[!htb]
  \centering
  \includegraphics[width=\linewidth]{theta_diff}
  \caption{Evolución de la diferencia angular $\theta$ entre las soluciones analítica y numérica a lo largo del tiempo para diferentes valores de ángulos iniciales.}
  \label{fig:diff}
\end{figure}

Por último, la Figura \ref{fig:tensiones_diff} muestra una comparativa entre la evolución de la tensión para los experimentos anteriores, tanto para los casos numéricos como analíticos. Como se puede comprobar, existe una pequeña discrepancia entre ambas (la cual es causada a su vez por la diferencia entre las posiciones $\theta$ obtenidas ya que la propia tensión es dependiente del ángulo del péndulo $T(\theta(t))$).

\begin{figure}[!htb]
  \centering
  \includegraphics[width=0.49\linewidth]{tension_diff}
  \includegraphics[width=0.49\linewidth]{tension_diff_approx}
  \caption{Evolución de la tensión $T$ a lo largo del tiempo para diversos valores iniciales del ángulo $\theta_0 = \{5.0, 10.0, 20.0, 40.0\}$ tanto para la solución numérica (izquierda) como analítica (derecha).}
  \label{fig:tensiones_diff}
\end{figure}

\section{Introducción de Rozamiento}

\begin{figure}[!htb]
  \centering
  \includegraphics[width=\linewidth]{alpha}
  \caption{Evolución de la amplitud $\theta$ con el tiempo $t$ para diferentes valores del término de rozamiento $\alpha$ con amplitud inicial $\theta_0=90~[deg]$. Valores calculados empleando la rutina de integración \mintinline{python}{odeint}.}
  \label{fig:alpha}
\end{figure}

\section{Conclusión}

% Can use something like this to put references on a page
% by themselves when using endfloat and the captionsoff option.
\ifCLASSOPTIONcaptionsoff
  \newpage
\fi

% trigger a \newpage just before the given reference
% number - used to balance the columns on the last page
% adjust value as needed - may need to be readjusted if
% the document is modified later
%\IEEEtriggeratref{8}
% The "triggered" command can be changed if desired:
%\IEEEtriggercmd{\enlargethispage{-5in}}

% references section

% can use a bibliography generated by BibTeX as a .bbl file
% BibTeX documentation can be easily obtained at:
% http://mirror.ctan.org/biblio/bibtex/contrib/doc/
% The IEEEtran BibTeX style support page is at:
% http://www.michaelshell.org/tex/ieeetran/bibtex/
%\bibliographystyle{IEEEtran}
% argument is your BibTeX string definitions and bibliography database(s)
%\bibliography{IEEEabrv,../bib/paper}
%
% <OR> manually copy in the resultant .bbl file
% set second argument of \begin to the number of references
% (used to reserve space for the reference number labels box)
\begin{thebibliography}{1}

\bibitem{IEEEhowto:kopka}
H.~Kopka and P.~W. Daly, \emph{A Guide to \LaTeX}, 3rd~ed.\hskip 1em plus
  0.5em minus 0.4em\relax Harlow, England: Addison-Wesley, 1999.

\end{thebibliography}

\end{document}
