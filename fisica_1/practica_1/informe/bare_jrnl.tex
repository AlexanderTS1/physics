%% bare_jrnl.tex
%% V1.4b
%% 2015/08/26
%% by Michael Shell
%% see http://www.michaelshell.org/
%% for current contact information.
%%
%% This is a skeleton file demonstrating the use of IEEEtran.cls
%% (requires IEEEtran.cls version 1.8b or later) with an IEEE
%% journal paper.
%%
%% Support sites:
%% http://www.michaelshell.org/tex/ieeetran/
%% http://www.ctan.org/pkg/ieeetran
%% and
%% http://www.ieee.org/

%%*************************************************************************
%% Legal Notice:
%% This code is offered as-is without any warranty either expressed or
%% implied; without even the implied warranty of MERCHANTABILITY or
%% FITNESS FOR A PARTICULAR PURPOSE! 
%% User assumes all risk.
%% In no event shall the IEEE or any contributor to this code be liable for
%% any damages or losses, including, but not limited to, incidental,
%% consequential, or any other damages, resulting from the use or misuse
%% of any information contained here.
%%
%% All comments are the opinions of their respective authors and are not
%% necessarily endorsed by the IEEE.
%%
%% This work is distributed under the LaTeX Project Public License (LPPL)
%% ( http://www.latex-project.org/ ) version 1.3, and may be freely used,
%% distributed and modified. A copy of the LPPL, version 1.3, is included
%% in the base LaTeX documentation of all distributions of LaTeX released
%% 2003/12/01 or later.
%% Retain all contribution notices and credits.
%% ** Modified files should be clearly indicated as such, including  **
%% ** renaming them and changing author support contact information. **
%%*************************************************************************


% *** Authors should verify (and, if needed, correct) their LaTeX system  ***
% *** with the testflow diagnostic prior to trusting their LaTeX platform ***
% *** with production work. The IEEE's font choices and paper sizes can   ***
% *** trigger bugs that do not appear when using other class files.       ***                          ***
% The testflow support page is at:
% http://www.michaelshell.org/tex/testflow/



\documentclass[journal]{IEEEtran}
%
% If IEEEtran.cls has not been installed into the LaTeX system files,
% manually specify the path to it like:
% \documentclass[journal]{../sty/IEEEtran}





% Some very useful LaTeX packages include:
% (uncomment the ones you want to load)
\usepackage[spanish]{babel}
\usepackage[utf8]{inputenc}
\usepackage{url}
\usepackage{minted}
\setminted{fontsize=\small,baselinestretch=1}
\usepackage{listings}
\usepackage{subcaption}
\renewcommand\listingscaption{Código}
\newenvironment{code}{\captionsetup{type=listing}}{\par\addvspace{\baselineskip}}
\usepackage{graphicx}
\usepackage{tikz}
\usetikzlibrary{scopes}
\usetikzlibrary{babel}

% *** MISC UTILITY PACKAGES ***
%
%\usepackage{ifpdf}
% Heiko Oberdiek's ifpdf.sty is very useful if you need conditional
% compilation based on whether the output is pdf or dvi.
% usage:
% \ifpdf
%   % pdf code
% \else
%   % dvi code
% \fi
% The latest version of ifpdf.sty can be obtained from:
% http://www.ctan.org/pkg/ifpdf
% Also, note that IEEEtran.cls V1.7 and later provides a builtin
% \ifCLASSINFOpdf conditional that works the same way.
% When switching from latex to pdflatex and vice-versa, the compiler may
% have to be run twice to clear warning/error messages.






% *** CITATION PACKAGES ***
%
%\usepackage{cite}
% cite.sty was written by Donald Arseneau
% V1.6 and later of IEEEtran pre-defines the format of the cite.sty package
% \cite{} output to follow that of the IEEE. Loading the cite package will
% result in citation numbers being automatically sorted and properly
% "compressed/ranged". e.g., [1], [9], [2], [7], [5], [6] without using
% cite.sty will become [1], [2], [5]--[7], [9] using cite.sty. cite.sty's
% \cite will automatically add leading space, if needed. Use cite.sty's
% noadjust option (cite.sty V3.8 and later) if you want to turn this off
% such as if a citation ever needs to be enclosed in parenthesis.
% cite.sty is already installed on most LaTeX systems. Be sure and use
% version 5.0 (2009-03-20) and later if using hyperref.sty.
% The latest version can be obtained at:
% http://www.ctan.org/pkg/cite
% The documentation is contained in the cite.sty file itself.






% *** GRAPHICS RELATED PACKAGES ***
%
\ifCLASSINFOpdf
  % \usepackage[pdftex]{graphicx}
  % declare the path(s) where your graphic files are
  % \graphicspath{{../pdf/}{../jpeg/}}
  % and their extensions so you won't have to specify these with
  % every instance of \includegraphics
  % \DeclareGraphicsExtensions{.pdf,.jpeg,.png}
\else
  % or other class option (dvipsone, dvipdf, if not using dvips). graphicx
  % will default to the driver specified in the system graphics.cfg if no
  % driver is specified.
  % \usepackage[dvips]{graphicx}
  % declare the path(s) where your graphic files are
  % \graphicspath{{../eps/}}
  % and their extensions so you won't have to specify these with
  % every instance of \includegraphics
  % \DeclareGraphicsExtensions{.eps}
\fi
% graphicx was written by David Carlisle and Sebastian Rahtz. It is
% required if you want graphics, photos, etc. graphicx.sty is already
% installed on most LaTeX systems. The latest version and documentation
% can be obtained at: 
% http://www.ctan.org/pkg/graphicx
% Another good source of documentation is "Using Imported Graphics in
% LaTeX2e" by Keith Reckdahl which can be found at:
% http://www.ctan.org/pkg/epslatex
%
% latex, and pdflatex in dvi mode, support graphics in encapsulated
% postscript (.eps) format. pdflatex in pdf mode supports graphics
% in .pdf, .jpeg, .png and .mps (metapost) formats. Users should ensure
% that all non-photo figures use a vector format (.eps, .pdf, .mps) and
% not a bitmapped formats (.jpeg, .png). The IEEE frowns on bitmapped formats
% which can result in "jaggedy"/blurry rendering of lines and letters as
% well as large increases in file sizes.
%
% You can find documentation about the pdfTeX application at:
% http://www.tug.org/applications/pdftex





% *** MATH PACKAGES ***
%
%\usepackage{amsmath}
% A popular package from the American Mathematical Society that provides
% many useful and powerful commands for dealing with mathematics.
%
% Note that the amsmath package sets \interdisplaylinepenalty to 10000
% thus preventing page breaks from occurring within multiline equations. Use:
%\interdisplaylinepenalty=2500
% after loading amsmath to restore such page breaks as IEEEtran.cls normally
% does. amsmath.sty is already installed on most LaTeX systems. The latest
% version and documentation can be obtained at:
% http://www.ctan.org/pkg/amsmath





% *** SPECIALIZED LIST PACKAGES ***
%
%\usepackage{algorithmic}
% algorithmic.sty was written by Peter Williams and Rogerio Brito.
% This package provides an algorithmic environment fo describing algorithms.
% You can use the algorithmic environment in-text or within a figure
% environment to provide for a floating algorithm. Do NOT use the algorithm
% floating environment provided by algorithm.sty (by the same authors) or
% algorithm2e.sty (by Christophe Fiorio) as the IEEE does not use dedicated
% algorithm float types and packages that provide these will not provide
% correct IEEE style captions. The latest version and documentation of
% algorithmic.sty can be obtained at:
% http://www.ctan.org/pkg/algorithms
% Also of interest may be the (relatively newer and more customizable)
% algorithmicx.sty package by Szasz Janos:
% http://www.ctan.org/pkg/algorithmicx




% *** ALIGNMENT PACKAGES ***
%
%\usepackage{array}
% Frank Mittelbach's and David Carlisle's array.sty patches and improves
% the standard LaTeX2e array and tabular environments to provide better
% appearance and additional user controls. As the default LaTeX2e table
% generation code is lacking to the point of almost being broken with
% respect to the quality of the end results, all users are strongly
% advised to use an enhanced (at the very least that provided by array.sty)
% set of table tools. array.sty is already installed on most systems. The
% latest version and documentation can be obtained at:
% http://www.ctan.org/pkg/array


% IEEEtran contains the IEEEeqnarray family of commands that can be used to
% generate multiline equations as well as matrices, tables, etc., of high
% quality.




% *** SUBFIGURE PACKAGES ***
%\ifCLASSOPTIONcompsoc
%  \usepackage[caption=false,font=normalsize,labelfont=sf,textfont=sf]{subfig}
%\else
%  \usepackage[caption=false,font=footnotesize]{subfig}
%\fi
% subfig.sty, written by Steven Douglas Cochran, is the modern replacement
% for subfigure.sty, the latter of which is no longer maintained and is
% incompatible with some LaTeX packages including fixltx2e. However,
% subfig.sty requires and automatically loads Axel Sommerfeldt's caption.sty
% which will override IEEEtran.cls' handling of captions and this will result
% in non-IEEE style figure/table captions. To prevent this problem, be sure
% and invoke subfig.sty's "caption=false" package option (available since
% subfig.sty version 1.3, 2005/06/28) as this is will preserve IEEEtran.cls
% handling of captions.
% Note that the Computer Society format requires a larger sans serif font
% than the serif footnote size font used in traditional IEEE formatting
% and thus the need to invoke different subfig.sty package options depending
% on whether compsoc mode has been enabled.
%
% The latest version and documentation of subfig.sty can be obtained at:
% http://www.ctan.org/pkg/subfig




% *** FLOAT PACKAGES ***
%
%\usepackage{fixltx2e}
% fixltx2e, the successor to the earlier fix2col.sty, was written by
% Frank Mittelbach and David Carlisle. This package corrects a few problems
% in the LaTeX2e kernel, the most notable of which is that in current
% LaTeX2e releases, the ordering of single and double column floats is not
% guaranteed to be preserved. Thus, an unpatched LaTeX2e can allow a
% single column figure to be placed prior to an earlier double column
% figure.
% Be aware that LaTeX2e kernels dated 2015 and later have fixltx2e.sty's
% corrections already built into the system in which case a warning will
% be issued if an attempt is made to load fixltx2e.sty as it is no longer
% needed.
% The latest version and documentation can be found at:
% http://www.ctan.org/pkg/fixltx2e


%\usepackage{stfloats}
% stfloats.sty was written by Sigitas Tolusis. This package gives LaTeX2e
% the ability to do double column floats at the bottom of the page as well
% as the top. (e.g., "\begin{figure*}[!b]" is not normally possible in
% LaTeX2e). It also provides a command:
%\fnbelowfloat
% to enable the placement of footnotes below bottom floats (the standard
% LaTeX2e kernel puts them above bottom floats). This is an invasive package
% which rewrites many portions of the LaTeX2e float routines. It may not work
% with other packages that modify the LaTeX2e float routines. The latest
% version and documentation can be obtained at:
% http://www.ctan.org/pkg/stfloats
% Do not use the stfloats baselinefloat ability as the IEEE does not allow
% \baselineskip to stretch. Authors submitting work to the IEEE should note
% that the IEEE rarely uses double column equations and that authors should try
% to avoid such use. Do not be tempted to use the cuted.sty or midfloat.sty
% packages (also by Sigitas Tolusis) as the IEEE does not format its papers in
% such ways.
% Do not attempt to use stfloats with fixltx2e as they are incompatible.
% Instead, use Morten Hogholm'a dblfloatfix which combines the features
% of both fixltx2e and stfloats:
%
% \usepackage{dblfloatfix}
% The latest version can be found at:
% http://www.ctan.org/pkg/dblfloatfix




%\ifCLASSOPTIONcaptionsoff
%  \usepackage[nomarkers]{endfloat}
% \let\MYoriglatexcaption\caption
% \renewcommand{\caption}[2][\relax]{\MYoriglatexcaption[#2]{#2}}
%\fi
% endfloat.sty was written by James Darrell McCauley, Jeff Goldberg and 
% Axel Sommerfeldt. This package may be useful when used in conjunction with 
% IEEEtran.cls'  captionsoff option. Some IEEE journals/societies require that
% submissions have lists of figures/tables at the end of the paper and that
% figures/tables without any captions are placed on a page by themselves at
% the end of the document. If needed, the draftcls IEEEtran class option or
% \CLASSINPUTbaselinestretch interface can be used to increase the line
% spacing as well. Be sure and use the nomarkers option of endfloat to
% prevent endfloat from "marking" where the figures would have been placed
% in the text. The two hack lines of code above are a slight modification of
% that suggested by in the endfloat docs (section 8.4.1) to ensure that
% the full captions always appear in the list of figures/tables - even if
% the user used the short optional argument of \caption[]{}.
% IEEE papers do not typically make use of \caption[]'s optional argument,
% so this should not be an issue. A similar trick can be used to disable
% captions of packages such as subfig.sty that lack options to turn off
% the subcaptions:
% For subfig.sty:
% \let\MYorigsubfloat\subfloat
% \renewcommand{\subfloat}[2][\relax]{\MYorigsubfloat[]{#2}}
% However, the above trick will not work if both optional arguments of
% the \subfloat command are used. Furthermore, there needs to be a
% description of each subfigure *somewhere* and endfloat does not add
% subfigure captions to its list of figures. Thus, the best approach is to
% avoid the use of subfigure captions (many IEEE journals avoid them anyway)
% and instead reference/explain all the subfigures within the main caption.
% The latest version of endfloat.sty and its documentation can obtained at:
% http://www.ctan.org/pkg/endfloat
%
% The IEEEtran \ifCLASSOPTIONcaptionsoff conditional can also be used
% later in the document, say, to conditionally put the References on a 
% page by themselves.




% *** PDF, URL AND HYPERLINK PACKAGES ***
%
%\usepackage{url}
% url.sty was written by Donald Arseneau. It provides better support for
% handling and breaking URLs. url.sty is already installed on most LaTeX
% systems. The latest version and documentation can be obtained at:
% http://www.ctan.org/pkg/url
% Basically, \url{my_url_here}.




% *** Do not adjust lengths that control margins, column widths, etc. ***
% *** Do not use packages that alter fonts (such as pslatex).         ***
% There should be no need to do such things with IEEEtran.cls V1.6 and later.
% (Unless specifically asked to do so by the journal or conference you plan
% to submit to, of course. )


% correct bad hyphenation here
\hyphenation{op-tical net-works semi-conduc-tor}


\begin{document}
%
% paper title
% Titles are generally capitalized except for words such as a, an, and, as,
% at, but, by, for, in, nor, of, on, or, the, to and up, which are usually
% not capitalized unless they are the first or last word of the title.
% Linebreaks \\ can be used within to get better formatting as desired.
% Do not put math or special symbols in the title.
\title{Estudio de La Caída de Una Gota}
%
%
% author names and IEEE memberships
% note positions of commas and nonbreaking spaces ( ~ ) LaTeX will not break
% a structure at a ~ so this keeps an author's name from being broken across
% two lines.
% use \thanks{} to gain access to the first footnote area
% a separate \thanks must be used for each paragraph as LaTeX2e's \thanks
% was not built to handle multiple paragraphs
%

\author{Alberto García García\\ (48718198-N)\\ \texttt{agg180@alu.ua.es} % <-this % stops a space
\thanks{}%
}

% note the % following the last \IEEEmembership and also \thanks - 
% these prevent an unwanted space from occurring between the last author name
% and the end of the author line. i.e., if you had this:
% 
% \author{....lastname \thanks{...} \thanks{...} }
%                     ^------------^------------^----Do not want these spaces!
%
% a space would be appended to the last name and could cause every name on that
% line to be shifted left slightly. This is one of those "LaTeX things". For
% instance, "\textbf{A} \textbf{B}" will typeset as "A B" not "AB". To get
% "AB" then you have to do: "\textbf{A}\textbf{B}"
% \thanks is no different in this regard, so shield the last } of each \thanks
% that ends a line with a % and do not let a space in before the next \thanks.
% Spaces after \IEEEmembership other than the last one are OK (and needed) as
% you are supposed to have spaces between the names. For what it is worth,
% this is a minor point as most people would not even notice if the said evil
% space somehow managed to creep in.



% The paper headers
\markboth{Física I -- Grado en Física -- 2018-2019}%
{}
% The only time the second header will appear is for the odd numbered pages
% after the title page when using the twoside option.
% 
% *** Note that you probably will NOT want to include the author's ***
% *** name in the headers of peer review papers.                   ***
% You can use \ifCLASSOPTIONpeerreview for conditional compilation here if
% you desire.




% If you want to put a publisher's ID mark on the page you can do it like
% this:
%\IEEEpubid{0000--0000/00\$00.00~\copyright~2015 IEEE}
% Remember, if you use this you must call \IEEEpubidadjcol in the second
% column for its text to clear the IEEEpubid mark.



% use for special paper notices
%\IEEEspecialpapernotice{(Invited Paper)}




% make the title area
\maketitle

% As a general rule, do not put math, special symbols or citations
% in the abstract or keywords.
\begin{abstract}
En esta primera práctica de la asignatura Física I del Grado en Física (curso académico 2018-2019) estudiaremos la caída de una gota de agua. Para ello consideraremos que la gota cae desde una nube alta y aproximaremos su forma mediante una esfera, de esta forma emplearemos la Ley de Stokes para expresar el rozamiento de la gota con el aire.

Este problema será resuelto de dos formas diferentes. Por una parte, estudiaremos la caída de forma analítica. Por otro lado, estudiaremos dicha caída de manera numérica, siendo esta segunda forma el objetivo principal de esta práctica. Para ello, discretizaremos el tiempo y resolveremos la ecuación diferencial suponiendo que en cada tramo o instante temporal $dt$ el movimiento transcurre con aceleración constante.

Una vez resuelto el problema tanto de forma numérica como analítica, calcularemos la velocidad con la que la gota impacta en el suelo y dispondremos en una gráfica la evolución de la velocidad y de la posición respecto al tiempo para comparar el resultado numérico con el analítico.

Además, obtendremos también la velocidad límite de la gota y experimentaremos con distintos tamaños y otros parámetros para comprobar la precisión de nuestra solución con las suposiciones anteriormente mencionadas respecto a las verdaderas velocidades alcanzadas por las gotas con modelos más cercanos a la realidad.

Por último, llevaremos a cabo una serie de experimentos cambiando la Ley de Stokes de forma que la velocidad $v$ quede expresada con una potencia mayor que $1$ con el objetivo de determinar qué ley o proporción es la más adecuada para representar la fricción de una gota de lluvia con el aire.

El código Python que implementa los modelos matemáticos así como las rutinas de visualización para la resolución de este ejercicio se adjunta con este informe y además puede ser consultado en el siguiente repositorio online \footnote{\url{https://github.com/Blitzman/physics}}.
\end{abstract}


% For peer review papers, you can put extra information on the cover
% page as needed:
% \ifCLASSOPTIONpeerreview
% \begin{center} \bfseries EDICS Category: 3-BBND \end{center}
% \fi
%
% For peerreview papers, this IEEEtran command inserts a page break and
% creates the second title. It will be ignored for other modes.
\IEEEpeerreviewmaketitle

\section{Introducción}

\IEEEPARstart{U}{na} gota de agua que cae verticalmente desde una nube puede modelarse como una esfera en el seno de un fluido que se mueve bajo la acción de diversas fuerzas: el peso, el empuje y una fuerza de rozamiento que puede formularse como proporcional a la velocidad según la Ley de Stokes.

En esta práctica simularemos la caída de una gota de radio $r$ que cae desde una altura $h$ y que se ve sometida a la fuerza ejercida tanto por su peso como por el rozamiento, omitiendo el empuje tal y como se muestra en la Figura \ref{fig:gota}.

\begin{figure}[!htb]
  \centering
  \begin{tikzpicture}[
      force/.style={>=latex,draw=blue,fill=blue},
      axis/.style={densely dashed,gray,font=\small},
      M/.style={rectangle,draw,fill=lightgray,minimum size=0.5cm,thin},
      m/.style={circle,draw=black,fill=lightgray,minimum size=0.3cm,thin},
      plane/.style={draw=black,fill=blue!10},
      string/.style={draw=red, thick},
      pulley/.style={thick},
  ]
\node[m] (m) {};
    \draw[axis,->] (m) -- ++(0,-2) node[left] {$y$};
    {[force,->]
        \draw [red](m.north) -- ++(0,1) node[above] {$F_d$};
        \draw (m.south) -- ++(0,-1) node[right] {$mg$};
    }
  \end{tikzpicture}
  \caption{Diagrama de cuerpo libre de caída de la gota.}
  \label{fig:gota}
\end{figure}

\section{Ley de Stokes}

La Ley de Stokes, en su forma general, hace referencia a la fuerza que se opone al movimiento y que es experimentada por cuerpos esféricos cuando se mueven en un fluido en un régimen laminar (números de Reynolds bajos). Esto significa que la Ley únicamente es válida para movimientos de partículas pequeñas a baja velocidad (mientras el régimen no sea turbulento). La ecuación de Stokes se define como:

\begin{equation}
  F_d = 6 \pi r \eta v
\end{equation}

, donde $\eta$ es la viscosidad del fluido, $r$ el radio de la esfera y $v$ la velocidad de la misma.

En nuestro caso, la gota se ve sometida a dos fuerzas: la gravitatoria (con sentido hacia el suelo) y la de rozamiento (en sentido opuesto). La fuerza de rozamiento se rige por la ecuación de Stokes previamente descrita. Como se puede comprobar, es una función de la velocidad $v$ de la partícula por lo que en un determinado instante de tiempo, ambas fuerzas se igualarán, produciendo entonces una aceleración nula y una velocidad constante (la denominada velocidad límite
$v_{lim}$).

\subsection{Solución Analítica}

La solución analítica a este problema podemos obtenerla a partir de las ecuaciones de Newton de la propia gota. Partiendo de la base

\begin{equation}
- F_r + P = m a~,
\end{equation}

donde $F_r$ es la ecuación de Stokes $6\pi r \eta v^{v_e}$ (siendo $v_e > 0$); podemos factorizar $D = 6\pi r \eta$ y entonces

\begin{equation}
  F_r = Dv = mg - ma~,
\end{equation}

y por lo tanto podemos despejar la aceleración como

\begin{equation}
  a = g - \displaystyle\frac{Dv}{m}~.
\end{equation}

Sabemos que $a = \displaystyle\frac{dv}{dt}$. Podemos entonces agrupar los términos de la siguiente forma

\begin{equation}
  a = g - \displaystyle\frac{Dv}{m} = \displaystyle\frac{mg - Dv}{m}
\end{equation}

para poder integrar a un lado y a otro:

\begin{equation}
  \displaystyle\frac{m}{mg - Dv} dv = dt~,
\end{equation}

por lo que

\begin{equation}
  m \int_{v_0}^v \displaystyle\frac{1}{mg - Dv}dv = t - t_0
\end{equation}
\begin{equation}
  - \displaystyle\frac{m}{D} ln(mg - Dv)|^v_{v_0} = \displaystyle\frac{m}{D}ln\left(\displaystyle\frac{mg - Dv_0}{mg - Dv}\right)~.
\end{equation}

Tomando que en $t_0 = 0~[s]$ la velocidad inicial $v_0 = 0~[m/s]$, entonces

\begin{equation}
  t = \displaystyle\frac{m}{D} ln \left(\displaystyle\frac{mg}{mg - Dv}\right)
\end{equation}

y por lo tanto, tomando exponentes

\begin{equation}
  e^(\displaystyle\frac{tD}{m}) = \displaystyle\frac{mg}{mg - Dv}
\end{equation}

llegamos a la solución analítica del problema:

\begin{equation}
  v(t) = \displaystyle\frac{mg}{D}(1 - e^{-(\displaystyle\frac{tD}{m})})~[m/s]
\end{equation}



\subsection{Solución Numérica}

Para implementar una solución numérica al problema del rozamiento y cálculo de la posición, velocidad y aceleración de la gota en cada instante discretizaremos el tiempo y supondremos que en cada tramo $dt$ el movimiento transcurre con aceleración constante. Así pues, la implementación se reduce a un bucle (cuyas condiciones iniciales son el tiempo $t = 0~[s]$ y la altura de la gota $y~[m]$) que iterará incrementando el instante de tiempo de acuerdo al intervalo $dt$ elegido hasta que la
gota impacte contra el suelo $y = 0~[m]$.

Así pues, en cada iteración se producen cuatro operaciones: (1) incremento del tiempo, (2) actualización de posición, (3) actualización de velocidad, (4) actualización de aceleración. El fragmento de Código \ref{code:bucle} muestra dichas operaciones.

\bigskip

\begin{code}
  \begin{minted}{python}
t_ += dt
y_ = update_position(y_, v_, dt)
v_ = update_velocity(v_, a_, dt)
a_ = update_acceleration(mass, v_, b, ve)
  \end{minted}
  \caption{Actualizaciones en bucle de simulación.}
  \label{code:bucle}
\end{code}

A lo largo del código, \mintinline{python}{t_} es el instante de tiempo actual en $[s]$, \mintinline{python}{dt} es el intervalo discreto de tiempo en $[s]$ y prefijado al inicio de la simulación, \mintinline{python}{y_}, \mintinline{python}{v_} y \mintinline{python}{a_} son la posición $[m]$, velocidad $[m/s]$ y la aceleración respectivamente $[m/s^2]$, \mintinline{python}{mass} es la masa de la gota $[kg]$ prefijada, \mintinline{python}{b} es la constante de la Ley de Stokes
$6\pi r\eta\rho$ y \mintinline{python}{ve} es la potencia de la velocidad en la fuera de rozamiento. Las funciones del bucle que actualizan la posición $y$, velocidad $v$ y aceleración $a$ se muestran en el fragmento de Código \ref{code:functions}.

\bigskip

\begin{code}
	\begin{minted}{python}
def update_acceleration(mass, velocity, b, ve):
  return (G - (b * velocity**ve / mass))

def update_velocity(velocity, acceleration, dt):
  return velocity + acceleration * dt

def update_position(position, velocity, dt):
  return position - velocity * dt
	\end{minted}
  \caption{Funciones de actualización.}
  \label{code:functions}
\end{code}

Como ya comentamos anteriormente, asumimos que en cada intervalo de tiempo el movimiento transcurre con aceleración y velocidad constante por lo que la posición puede calcularse de forma sencilla como:

\begin{equation}
  y(t) = y(t-1) - v(t) \cdot dt~.
\end{equation}

De igual manera podemos proceder con la velocidad:

\begin{equation}
  v(t) = v(t-1) + a(t) \cdot dt~.
\end{equation}

La aceleración por su parte se obtiene teniendo en cuenta las dos fuerzas que actúan (la gravedad y el rozamiento) así como la masa de la gota:

\begin{equation}
  a(t) = \displaystyle\frac{m\cdot g - b \cdot v^{v_e}}{m} = g - \displaystyle\frac{b\cdot v^{v_e}}{m}~.
\end{equation}

\subsection{Solución Analítica vs. Numérica}

Una vez implementadas ambas soluciones, ejecutamos la simulación para realizar una comparativa de los resultados proporcionados por ambas. Para esta comparativa elegimos un conjunto de parámetros que nos permitiera observar el alcance de la velocidad límite para gotas de agua de diferente tamaño. Aunque habitualmente las gotas de agua caen por encima de los $2000~[m]$ de altura, para los tamaños considerados como habituales \cite{Laws}\cite{Corbert} (alrededor de los $0.25~[mm]$ de radio para
las gotas más pequeñas), la
velocidad límite se alcanza mucho antes de haber recorrido dicho espacio. En nuestro caso con una altura $h = 100~[m]$ para gotas de radio $r = \{0.15, 0.25, 0.30\}~[mm]$ podemos alcanzar la velocidad límite en todos los casos por lo que elegiremos esta altura para reducir el tiempo de experimentación. Adicionalmente, cabe tener en cuenta los valores para la viscosidad del aire $\eta = 18\cdot10^{-16}~[Ns/m^2]$ y la densidad del agua $\rho = 1000~[kg/m^3]$. Los resultados de la simulación ejecutada con un intervalo de tiempo $dt = 0.001~[s]$ y con $t_0 =
0~[s]$ se muestran en la Figura \ref{fig:analitica} y se resumen en la Tabla \ref{table:analitica}. Cabe destacar que ignoramos el efecto de la altura sobre la fuerza gravitatoria y consideramos en todo momento $g = 9.8~[m/s^2]$.

\begin{figure}[!htb]
	\centering
	\includegraphics[width=\linewidth]{graficas_analitica}
  \caption{Gráficas de posición y velocidad que muestran la evolución de dichas componentes a lo largo del tiempo de simulación para diferentes tamaños de gota tanto para la solución numérica como para la analítica (A).}
  \label{fig:analitica}
\end{figure}

\begin{table}[!htb]
  \centering
	\begin{tabular}{ccccc}
    Radio $[mm]$ & \multicolumn{2}{c}{Velocidad Límite $[m/s]$} & \multicolumn{2}{c}{Tiempo de Impacto $[s]$}\\
    & Numérica & Analítica & Numérica & Analítica\\
		\hline
    $0.15$ & 2.72 & 2.72 & 183.95 & 183.95 \\
    $0.25$ & 7.56 & 7.56 & 66.89 & 66.89 \\
    $0.30$ & 10.89 & 10.89 & 47.03 & 47.03 \\
	\end{tabular}
  \caption{Comparativa de resultados obtenidos mediante la solución analítica y la solución numérica para gotas de radio $r=\{0.15, 0.25, 0.30\}$.}
  \label{table:analitica}
\end{table}

Como podemos comprobar, no existe ninguna diferencia significativa en los resultados finales de cualquiera de las dos aproximaciones tanto analítica como numérica.

\section{Experimentación}

Una vez expuestas las soluciones tanto analítica como numérica y comparados los resultados obtenidos, procederemos a realizar una experimentación más extensa sobre la implementación numérica. Este conjunto de experimentos consistirá por un lado en la comparación de la velocidad límite obtenida para gotas de diferentes tamaños y por otro en el estudio del efecto de la potencia de $v$ en la Ley de Stokes.

En ambos casos emplearemos el mismo juego de parámetros para todos los factores externos que no tienen que ver ni con la potencia de la velocidad ni con el tamaño de la gota: la densidad del agua $\rho = 1000~[kg/m^3]$, la viscosidad del aire $\eta = 18 \cdot 10^{-6}~[N\cdot s/m^2]$ a $20$ grados centígrados, la altura desde la cual cae la gota $h = 100~[m]$ y el intervalo o diferencial de tiempo $dt = 0.001~[s]$.

\subsection{Radio de la Gota}

En este primer conjunto de experimentos variaremos el tamaño de la gota de agua, es decir, el radio de la misma, lo cual afectará tanto a la fuerza de rozamiento (ya que recordemos que el radio $r$ es uno de sus componentes) como a la aceleración (puesto que al variar el radio también cambiará el volumen y por lo tanto la masa de la gota). Para estas pruebas hemos utilizado una potencia de $v$ de $1$ y los valores del radio de la gota $r = [0.10, 0.15, 0.20, 0.25, 0.30]~[mm]$.

\begin{table}[!htb]
	\begin{tabular}{cccc}
		Radio $[mm]$ & Velocidad Límite $[m/s]$ & Tiempo de Impacto $[s]$\\
		\hline
		$0.10$ & 1.21 & 82.77\\
		$0.15$ & 2.72 & 37.01\\
		$0.20$ & 4.84 & 21.16\\
		$0.25$ & 7.56 & 14.00\\
		$0.30$ & 10.89 & 10.30\\
	\end{tabular}
  \caption{Resumen de experimentos de radio mostrando la velocidad límite de cada gota y el tiempo necesario para lograr el impacto contra el suelo desde el inicio de la simulación.}
  \label{table:radio}
\end{table}

\newpage

\begin{figure}[!htb]
	\centering
	\includegraphics[width=1\linewidth]{graficas_radio}
	\caption{Gráficas de posición, velocidad y aceleración que muestran la evolución de dichas componentes a lo largo del tiempo de simulación para diferentes tamaños de gota.}
  \label{fig:radio}
\end{figure}

\newpage

Como muestran las gráficas de la Figura \ref{fig:radio} (resumidos sus valores en la Tabla \ref{table:radio}), la velocidad límite depende en gran medida del tamaño de la gota, siendo mayor a medida que el radio de la gota aumenta. En comparación a la velocidad límite de las gotas en la realidad se observa una notable diferencia. Por ejemplo, en el caso de la gota de $0.25~[mm
]$ de radio ($0.50~[mm]$ de radio), su velocidad límite se establece en $7.56~[m/s]$ según nuestra solución, mientras que otros estudios \cite{Holladay} \cite{Spilhaus} \cite{Beard} sugieren una velocidad de alrededor de entre $2~[m/s]$ y $3~[m/s]$.

\subsection{Potencia de $v$}

En esta segunda aprte de los experimentos cambiaremos la Ley de Stokes de forma que la potencia de $v$ tome diferentes valores $v_e = [1, 2, 3, 4]$ con el fin de determinar cuál es la Ley más adecuada para representar la fricción de una gota de lluvia con el aire. Para estas pruebas utilizamos el mismo juego de parámetros previamente descrito y una gota de lluvia de radio $r = 0.25~[mm]$. La Figura \ref{fig:ve} muestra la evolución de la posición, velocidad y aceleración para
dichos experimentos.

\begin{figure}[!htb]
	\centering
	\includegraphics[width=\linewidth]{ve}
  \label{fig:ve}
  \caption{Gráficas de posición, velocidad y aceleración que muestran la evolución de dichas componentes a lo largo del tiempo de simulación para diferentes valores del exponente de la velocidad en el término de rozamiento $ve$ con una gota de radio $0.25~[mm]$.}
\end{figure}

Podemos observar los diferentes valores de la velocidad límite obtenidos para las diversas potencias: en el caso que ya estudiamos anteriormente $v_e = 1$ la velocidad límite es de $7.56~[m/s]$, para $v_e = 2$ la velocidad es $2.74~[m/s]$, para $v_e = 3$ es $1.96~[m/s]$ y para $v_e = 4$ se obtiene una velocidad límite de $1.66~[m/s]$. Según los estudios antes mencionados, la velocidad límite de una gota de $0.50~[mm]$ de diámetro se aproxima a $2.50~[m/s]$. Esta
correspondencia implica que un exponente cuadrático refleja mejor el rozamiento producido, hipótesis que concuerda con el hecho de que la Ley de Stokes se cumpla únicamente para números de Reynold bajos (es decir, mientras que la velocidad baja con una esfera de radio reducido y el flujo sea laminar) y que para el resto de situaciones el término de rozamiento sea proporcional al cuadrado de la velocidad.

\section{Conclusión}

En esta primera práctica de la asignatura hemos presentado un estudio completo de la caída de una gota de agua. Para ello, hemos implementado una simulación en Python capaz de resolver el problema de dos formas teniendo en cuenta las ecuaciones del movimiento de la gota: analítica y numérica. A lo largo de la experimentación hemos validado ambas soluciones, mostrando que no existe discrepancia entre ellas. En una tanda de experimentación adicional mostramos el efecto que tiene el
radio de la gota en su velocidad de caída así. Por último, demostramos empíricamente cuál es el exponente más adecuado para la velocidad en el término de rozamiento.

El código Python que implementa los modelos matemáticos así como las rutinas de visualización para la resolución de este ejercicio se adjunta con este informe y además puede ser consultado en el siguiente repositorio online\footnote{\url{https://github.com/Blitzman/physics}}.

% Can use something like this to put references on a page
% by themselves when using endfloat and the captionsoff option.
\ifCLASSOPTIONcaptionsoff
  \newpage
\fi

% trigger a \newpage just before the given reference
% number - used to balance the columns on the last page
% adjust value as needed - may need to be readjusted if
% the document is modified later
%\IEEEtriggeratref{8}
% The "triggered" command can be changed if desired:
%\IEEEtriggercmd{\enlargethispage{-5in}}

% references section

% can use a bibliography generated by BibTeX as a .bbl file
% BibTeX documentation can be easily obtained at:
% http://mirror.ctan.org/biblio/bibtex/contrib/doc/
% The IEEEtran BibTeX style support page is at:
% http://www.michaelshell.org/tex/ieeetran/bibtex/
%\bibliographystyle{IEEEtran}
% argument is your BibTeX string definitions and bibliography database(s)
%\bibliography{IEEEabrv,../bib/paper}
%
% <OR> manually copy in the resultant .bbl file
% set second argument of \begin to the number of references
% (used to reserve space for the reference number labels box)
\begin{thebibliography}{1}

\bibitem{Corbert}
  J.H.~Corbert, \emph{Physical Geography Manual}, 1974. 5th ed. N.p.: Kendall/Hunt, 2003.
\bibitem{Laws}
  J.O.~Laws, \emph{ Measurements of the fall‐velocity of water‐drops and raindrops}. Eos, Transactions American Geophysical Union (1947), 22(3), 709-721.
\bibitem{Beard}
  K.V.~Beard, \emph{Terminal Velocity and Shape of Cloud and Precipitation Drops}, Journal of the Atmospheric Sciences (May 1976): 851-864.
\bibitem{Spilhaus}
  A.F.~Spilhaus, \emph{Raindrop Size, Shape, and Falling Speed}, Journal of Meteorology. 5 (June 1948): 108-110.
\bibitem{Holladay}
  A.~Holladay, \emph{Falling Raindrops Hit 5 to 20 mph speeds}, Wonderquest.

\end{thebibliography}

\end{document}
